\documentclass[
	12pt,
	a4paper,
	bibtotoc,
	cleardoubleempty,
	idxtotoc,
	ngerman,
	openright,
	final,
	listof=nochaptergap,
	]{scrbook}




\usepackage[T1]{fontenc}
\usepackage[utf8]{inputenc}
\usepackage[printonlyused]{acronym}
\usepackage{tabularx}
\usepackage{listings}
\usepackage{quoting}

% ##################################################
% Unterstuetzung fuer die deutsche Sprache
% ##################################################
%\usepackage{ngerman}
\usepackage[ngerman]{babel}

% ##################################################
% Dokumentvariablen
% ##################################################

% Persoenliche Daten
\newcommand{\docNachname}{Israel}
\newcommand{\docVorname}{Marco}
\newcommand{\docStrasse}{Am Wickenkamp 38}
\newcommand{\docOrt}{Stemwede}
\newcommand{\docPlz}{32351}
\newcommand{\docEmail}{Marco-Israel-Consulting@gmx.de}
\newcommand{\docMatrikelnummer}{242814}

% Dokumentdaten
\newcommand{\docTitle}{Embedded Linux}
\newcommand{\docUntertitle}{danke Yocto und OpenEmbedded}
%\newcommand{\docUntertitle}{} % Kein Untertitel
% Arten der Arbeit: Bachelorthesis, Masterthesis, Seminararbeit, Diplomarbeit
\newcommand{\docArtDerArbeit}{}
%Studiengaenge: Allgemeine Informatik Bachelor, Computer Networking Bachelor,
% Software-Produktmanagement Bachelor, Advanced Computer Scinece Master
\newcommand{\docStudiengang}{}
\newcommand{\docAbgabedatum}{Dezember 2019}
\newcommand{\docErsterReferent}{}
%\newcommand{\docZweiterReferent}{-} % Wenn es nur einen Betreuer gibt
%\newcommand{\docZweiterReferent}{ZWEITER REFERENT}

% ##################################################
% Allgemeine Pakete
% ##################################################

% Abbildungen einbinden
\usepackage{graphicx}

% Zusaetsliche Sonderzeichen
\usepackage{dingbat}

% Farben
%\usepackage{or}
%\usepackage[usenames,dvipsnames,svgnames,table]{xcolor}

% Maskierung von URLs und Dateipfaden
\usepackage{url}

% Deutsche Anfuehrungszeichen
\usepackage[babel, german=quotes]{csquotes}

% Pakte zur Index-Erstellung (Schlagwortverzeichnis)
\usepackage{index}
\makeindex

% Ipsum Lorem
% Paket wird nur für das Beispiel gebraucht und kann gelöscht werden
\usepackage{lipsum}

% ##################################################
% Seitenformatierung
% ##################################################
\usepackage[
	portrait,
	bindingoffset=1.5cm,
	inner=2.5cm,
	outer=2.5cm,
	top=3cm,
	bottom=2cm,
	includeheadfoot
	]{geometry}

% ##################################################
% Kopf- und Fusszeile
% ##################################################

\usepackage{fancyhdr}

\pagestyle{fancy}
\fancyhf{}
\fancyhead[EL,OR]{\sffamily\thepage}
\fancyhead[ER,OL]{\sffamily\leftmark}

\fancypagestyle{headings}{}

\fancypagestyle{plain}{}

\fancypagestyle{empty}{
  \fancyhf{}
  \renewcommand{\headrulewidth}{0pt}
}

%Kein "Kapitel # NAME" in der Kopfzeile
\renewcommand{\chaptermark}[1]{
	\markboth{#1}{}
   	\markboth{\thechapter.\ #1}{}
}

% ##################################################
% Schriften
% ##################################################

% Stdandardschrift festlegen
\renewcommand{\familydefault}{\sfdefault}

% Standard Zeilenabstand: 1,5 zeilig
\usepackage{setspace}
\onehalfspacing

% Schriftgroessen festlegen
\addtokomafont{chapter}{\sffamily\large\bfseries}
\addtokomafont{section}{\sffamily\normalsize\bfseries}
\addtokomafont{subsection}{\sffamily\normalsize\mdseries}
\addtokomafont{caption}{\sffamily\normalsize\mdseries}

% ##################################################
% Inhaltsverzeichnis / Allgemeine Verzeichniseinstellungen
% ##################################################

\usepackage{tocloft}

% Punkte auch bei Kapiteln
\renewcommand{\cftchapdotsep}{3}
\renewcommand{\cftdotsep}{3}

% Schriftart und -groesse im Inhaltsverzeichnis anpassen
\renewcommand{\cftchapfont}{\sffamily\normalsize}
\renewcommand{\cftsecfont}{\sffamily\normalsize}
\renewcommand{\cftsubsecfont}{\sffamily\normalsize}
\renewcommand{\cftchappagefont}{\sffamily\normalsize}
\renewcommand{\cftsecpagefont}{\sffamily\normalsize}
\renewcommand{\cftsubsecpagefont}{\sffamily\normalsize}

%Zeilenabstand in den Verzeichnissen einstellen
\setlength{\cftparskip}{.5\baselineskip}
\setlength{\cftbeforechapskip}{.1\baselineskip}

% ##################################################
% Abbildungsverzeichnis und Abbildungen
% ##################################################

\usepackage{caption}

\usepackage{wrapfig}

% Nummerierung von Abbildungen
\renewcommand{\thefigure}{\arabic{figure}}

% Abbildungsverzeichnis anpassen
\renewcommand{\cftfigpresnum}{Abbildung }
\renewcommand{\cftfigaftersnum}{:}

% Breite des Nummerierungsbereiches [Abbildung 1:]
\newlength{\figureLength}
\settowidth{\figureLength}{\bfseries\cftfigpresnum\cftfigaftersnum}
\setlength{\cftfignumwidth}{\figureLength}
\setlength{\cftfigindent}{0cm}

% Schriftart anpassen
\renewcommand\cftfigfont{\sffamily}
\renewcommand\cftfigpagefont{\sffamily}

% ##################################################
% Tabellenverzeichnis und Tabellen
% ##################################################

% Nummerierung von Tabellen
\renewcommand{\thetable}{\arabic{table}}

% Tabellenverzeichnis anpassen
\renewcommand{\cfttabpresnum}{Tabelle }
\renewcommand{\cfttabaftersnum}{:}

% Breite des Nummerierungsbereiches [Abbildung 1:]
\newlength{\tableLength}
\settowidth{\tableLength}{\bfseries\cfttabpresnum\cfttabaftersnum}
\setlength{\cfttabnumwidth}{\tableLength}
\setlength{\cfttabindent}{0cm}

%Schriftart anpassen
\renewcommand\cfttabfont{\sffamily}
\renewcommand\cfttabpagefont{\sffamily}

% Unterdrueckung von vertikalen Linien
\usepackage{booktabs}

% ##################################################
% Listings (Quellcode)
% ##################################################

\usepackage{listings}
\lstset{
	language=java,
	backgroundcolor=\color{white},
	breaklines=true,
	prebreak={\carriagereturn},
 	breakautoindent=true,
 	numbers=left,
 	numberstyle=\tiny,
 	stepnumber=2,
 	numbersep=5pt,
 	keywordstyle=\color{blue},
   	commentstyle=\color{green},
   	stringstyle=\color{gray}
}

% ##################################################
% Theoreme
% ##################################################

% Umgebung fuer Beispiele
\newtheorem{beispiel}{Beispiel}

% Umgebung fuer These
\newtheorem{these}{These}

% Umgebung fuer Definitionen
\newtheorem{definition}{Definition}

% ##################################################
% Literaturverzeichnis
% ##################################################

\usepackage{bibgerm}

% ##################################################
% PDF / Dokumenteninternelinks
% ##################################################

\usepackage[
	colorlinks=false,
   	linkcolor=red,
   	citecolor=green,
  	filecolor=magenta,
	urlcolor=cyan,
    bookmarks=true,
    bookmarksopen=true,
    bookmarksopenlevel=3,
    bookmarksnumbered,
    plainpages=false,
    pdfpagelabels=true,
    hyperfootnotes,
    pdftitle ={\docTitle},
    pdfauthor={\docVorname~\docNachname},
    pdfcreator={\docVorname~\docNachname}]{hyperref}

% ##################################################
% Glossrry
% Muss nach "hyperref"
% ##################################################
    \usepackage{hyperref}
    \usepackage{glossaries}
    \glossarystyle{listhypergroup}
    \makeglossaries
%    \loadglsentries{content/glossary} %Glossary in externer Datei...
    % % % % % % % % % % % % % % % % %Glossar
%\usepackage{ngerman}
%\usepackage[ngerman]{babel}

\newglossaryentry{NFS}{
name=NFS,
description={ Das \acf{NFS}, (auch Network File Service) ist ein
    Netzwerkprotokoll, das den Zugriff auf Dateien über ein Netzwerk ermöglicht.
    Dabei werden die Dateien nicht wie z. B. bei \ac{FTP} oder \ac{TFTP}
    übertragen, sondern die Benutzer können auf Dateien, die sich auf einem
    entfernten Rechner befinden, so zugreifen, als ob sie auf ihrer lokalen
    Festplatte abgespeichert wären. Unter Unix Betriebssystemen lassen sich
    diese Netzwerkfreigaben direkt in das Dateisystem einhängen entsprechend wie
z.B. Festplatten. }
}


\newglossaryentry{TFTP}{
name=TFTP,
description={Das \acf{TFTP} ist ein im
    Fumktionsumfang stark vereinfachtes \acl{FTP} und reduziertes Protokoll, das
lediglich das senden und das empfangen von Dateien ermöglicht.}
}

\newglossaryentry{GIT}{
name=GIT,
description={ Git ist eine freie Software zur verteilten Versionsverwaltung von
    Dateien, die durch Linus Torvalds initiiert wurde. Es arbeiten im gegensatz
    zu anderen Versionsverwaltungen mit lokalen als auch externen Reposetories.
}
}

\newglossaryentry{Dockerfile}{
name=Dockerfile,
description={Ein Dockerfile beschreibt wie ein \gls{Docker} Container (Image)
    gebaut werden soll, genauer welche Tools und Konfigurationen der
    Container enthalten soll.}
}


\newglossaryentry{Yocto}{
name=Yocto,
description={
Das Yocto Project bezeichnet eine Community Gruppe, welche
sich zur Aufgabe gemacht hat, dass erstellen von Linux Distributionen für
eingebettet System zu vereinfachen. Die Yocto community stellt Meta-Daten (Recipes) bereit, um ein minimalistisches
Betriebssystem mit grundlegenden  Linux \glspl{Tool} unter der Virtualisierungsumgebung
\textit{\gls{QEMU}} für verschiedene Architekturen starten zu können.
Das Minimalisten Betriebssystem der Yocto Community wird \textit{\gls{Poky}}
genannt. Zudem Pflegt es zusammen mit der \gls{OpenEmbedded} community das Buildsystem
\textit{bitbake}.  }
}


\newglossaryentry{QEMU}{
name=QEMU,
description={QEMU ist (von englisch „Quick Emulator“) ist eine freie
    Virtualisierungssoftware, die die gesamte Hardware mitsamt
    Prozessorinstruktionen einer einer Zielhardware emuliert. Dabei ist es sehr
Recourcenschonend und performant.}
}

\newglossaryentry{Wrapper}{
name=Wrapper,
description={ Ein Wrapper ist ein Sofwareinterface das Interaktion mit einem
    anderen Softwarepaket kapselt und i.d.R. nach außen hin vereinfacht.
}
}



\newglossaryentry{OpenEmbedded}{
name=OpenEmbedded,
description={ Eine Community Gruppe welche eine Softwareware Buildumgebung
namens Bitbake pflegt und Metatdaten für dieses Buildsystem bereitstellt,
welche von dieser Buildumgebung verwendet wird um Softwarepakete zu übersetzen}
}

\newglossaryentry{Recipes}{
name=Recipies,
description={Rezepte oder Anleitungen}
}


\newglossaryentry{Metadaten}{
name=Metadaten,
description={Metadaten oder Metainformationen sind Informationen über andere
Daten}
}

\newglossaryentry{Docker}{
name=Docker,
description={Docker ist eine Software zur virtualisierung von einzelnen
    Anwendungen}
}

\newglossaryentry{SDK}{
    name=SDK,
    description={Ein \ac{SDK} ist eine gekapselte bzw.
vordefinierte Umgebung zur Entwicklung von Software\-komponenten. Die Umgebung
stellt (vorkonfigurierte) Sammlungen von ausgewählten Programmier\-werkzeugen
oder auch Liberias bereit, die zur Entwicklung für eine Zielplattform benötigt
werden.}
}

\newglossaryentry{rootfs}{
    name=rootfs,
    description={Unter Linux wird das urspüngliche  \textit{rootfs} als der Ort
        bezeichnet, von  dem \textbf{ausgehend} alle weiteren Verzeichnissbäume
        eingehängt sind/werden. Beispielweise liegen direkt unterhalb des rootfs
        die Ordner /home, /boot /root oder /bin.}
}

\newglossaryentry{Workflow}{
    name=Workflow,
    description={Ein Wokflow ist ein Arbeitsablauf und beschreibt Schritte in
        ihrer Reihenfolge die nötig sind, um eine Aufgabe, Arbeitspaket oder
        etwa eine Anweisung zu erfüllen. Dabei sind die Arbeitsschritte häufig
        wiederkehrend in anderen Arbeitsabläufen.
        } }

\newglossaryentry{dummy}{
    name=dummy,
    description=dummy{
        } }

\newglossaryentry{Poky}{
    name=Poky,
    description={Minimalistisches Betriebssystem der \gls{Yocto} Community das
        mitsamt seiner Tools in einer \gls{QEMU} virtuellen Maschine lauffähig
        ist.
        } }

\newglossaryentry{Distribution}{
    name=Distribution,
    plural=Distributionen,
    description={Als Distribution bezeichnet man eine Zusammenstellung von
        (Software-) Paketen,Versionen,  Konfigurationen, Einstellungen, usw.
        die als Gesamtpaket veröffentlicht sind oder werden und in sich ohne
        weiteres Zutun eine definierte Aufgabe erfüllen. Beispielsweise ist
        ein Betriebssystem eine solche Zusammenstellung das ohne weiters Zutun
        für einen Satz von Anwendungsfällen genutzt werden kann.
        } }


\newacronym{DTB}{DTB}{Device Tree Blob} \newacronym{BSP}{BSP}{Board Support
    Packed}


%
%
%\newglossaryentry{Mircosoft Office}{
%name=Mircosoft Office,
%description={Ein Softwarepaket, welches verschiedene Programme für unterschiedliche Aufgaben enthält. Weiteres ist unter
%\cite{ms1}}
%}
%
%\newglossaryentry{Portables Job Definition Format}{
%name=Portables Job Definition Format,
%description={
%\ac{PJDF} ist ein von Adobe entwickeltes Format zur Speicherung technischer Produktions- und Auftragsdaten auf Basis der PDF Format Syntax. \cite{pjdf1}}
%}
%
%\newglossaryentry{32-Bit Architektur}{
%name=32-Bit Architektur,
%plural=32-Bit Architekturen,
%description={32-Bit kennzeichnet eine PC bzw. Prozessor oder Platinen Architektur. Sie kennzeichnet die Breite des Adressbusses und gibt somit vor, wie viel Speicher (adressierbare Blöcke oder Einheiten mit jeweils einem Byte) eine Architektur ansprechen und verarbeiten kann. Mit einem 32-Bit-Adressbus lassen sich maximal 2e32 Byte adressieren. Dies sind 232 Speicherstellen mit jeweils einem Byte. Oder umgerechnet 4 GiB. Die Architekturform hat vor allem Einfluss auf die Prozessor- und Busarchitektur. Diese hängen unmittelbar zusammen und sind in Bezug auf Ihre Geschwindigkeit abhängig von der Architekturform (in diesem Fall beschränkt auf 32 Bit). Hieraus ergibt sich auch der Begriff \glqq 32-Bit Betriebssystem\grqq\, welches auf die Prozessor- und Busarchitektur aufsetzt. Einem 32-Bit Betriebssystem ist es somit auch nur möglich, Max. 4 GB (bzw. GiB) Arbeitsspeicher anzusprechen und zu verarbeiten (teilweise sogar weniger).  }}
%
%\newglossaryentry{32-Bit Betriebssystem}{
%name=32-Bit Betriebssystem,
%plural=32-Bit Betriebssysteme,
%description={Siehe \gls{32-Bit Architektur}}
%}
%
%\newglossaryentry{Datentyp}{
%name=Datentyp,
%plural=Datentypen,
%description={Datentyp beschreibt den Wertebereich von \glspl{Variable}. Sie werden daher auch als Werteart oder Datenart beschrieben. Sie sind gekennzeichnet durch einen Wertebereich, einen Name, einen Geltungsbereich, einer Konstruktionsregel, und einer gültigen (zulässige) Menge von Operationen über diese Wertemenge. Es wird zwischen numerische, boolesche, alphanumerische, abstrakte, elementare, zusammengesetzte und benutzerdefinierte Datentypen unterschieden. }
%}
%
%
%\newglossaryentry{Variable}{
%name=Variable,
%plural=Variablen,
%description={Eine Variable ist in der Informatik ein Platzhalter. Er ist gekennzeichnet durch einen Namen und häufig durch einen Datentyp. Variablen werden (temporäre) Werte zugeordnet. Diese können verändert, überschrieben oder gelöscht werden. Sie dienen häufig als (Zwischen-) Speicher, z.B. für eine Berechnung. }
%}
%
%\newglossaryentry{Konversationen}{
%name=Konversationen,
%description={Einigung, Normung oder Übereinkunft z.B. über einen Zeichensatzes. Zu dieser Einigung ist man gemeinsam in einem Gespräch (Konversation) gekommen. Diese Einigung ist häufig zu einer De-facto- oder Quasi-Vorgabe geworden.}
%}
%\newglossaryentry{Forms Data Format}{
%name=Forms Data Format,
%description={\ac{FDF} ist ein Format zur (Serverseitigen-) Verarbeitung der Formulardaten innerhalb eines PDF Dokumenten \cite{fdf1} \cite{fdf3} \cite{fdf2}}
%}
%\newglossaryentry{String}{
%name=String,
%plural=Strings,
%description={Ein String bezeichnet eine Zeichenkette in der Informatik. Sie repräsentiert z.B. ein Wort oder einen Satz.}
%}
%\newglossaryentry{Hypertext Transfer Protocol}{
%name=Hypertext Transfer Protocol,
%description={Das Hypertext Transfer Protocol (HTTP) ist ein Protokoll zur Übertragung von Daten auf der Anwendungsschicht in Rechnernetzen. Meistens wird es genutzt Hypertext-Dokumente (Webseiten) aus dem World Wide Web (WWW) in einen Webbrowser zu laden. Das Protokoll ist Zustandslos}
%}
%\newglossaryentry{Backdoor}{
%name=Backdoor,
%description={Hintertür für Angreifer, damit diese sich zu dem Rechner verbinden und diesen eindringen können.}
%}
%\newglossaryentry{Owner}{
%name=Owner,
%description={Als \glqq Owner\grqq\ wird der Besitzer von etwas bezeichnet. Z.b. der Besitzer (Owner) einer Datei oder eines Dokumentes. Er ist oft gleichzusetzen mit einem Administrator. Der Owner besitzt i.d.R. die vollen Zugriffsrechte, inklusive Rechte zum Verändern der Inhalte oder Löschen der kompletten Datei.}
%}
%\newglossaryentry{Pubic-Private Key Verfahren}{
%name=Pubic-Private Key Verfahren,
%description={Ein Public-Privat-Key Verschlüsselungsverfahren (kurz Public-Key Verfahren) ein kryptographisches Schutz Verfahren. Bei diesem wird mit einem veröffentlichten (öffentlicher) Schlüssel eine Klartextdatei in eine verschlüsselte Datei umgewandelt. Die Datei kann nur wieder durch einen zweiten, geheimen Schlüssel in Klartextform umgewandelt werden. Dieses wird asymmetrisches Verschlüsselungsverfahren bezeichnet, da zwei unterschiedliche Schlüssel (ein öffentlicher und ein geheimer) benötigt werden um ein Dokument zu verschlüsseln und weder zu entschlüsseln.}
%}
%
%
%%\newglossaryentry{XRef Sektion}{
%%name=XRef Sektion,
%%plural=XRef Sektionen,
%%description={Eine Cross Referenz Sektion (XRef Sektion) beschreibt den Bereich einer Cross Referenz Tabelle oder des Cross Referenz Stream Objektes in einem PDF Dokument}
%%}
%
%\newglossaryentry{User}{
%name=User,
%description={Als \glqq User\grqq\ wird ein normaler Benutzer bezeichnet. Er besitzt häufig weniger, bzw. eingeschränkte (Zugriffs-) Rechte, etwa auf eine Datei oder Ordner. (Im Gegensatz zu einem Owner oder Administrator.) Häufig darf er etwa Dateien nur lesen oder eingeschränkt bearbeiten und verwenden.}
%}
%\newglossaryentry{Adobe Reader}{
%name=Adobe Reader,
%description={Software zu Anzeigen von PDF Dateien. Es stammt vom Hersteller Adobe, welcher auch das \ac{PDF} Format entworfen hat.}
%}
%\newglossaryentry{Layout}{
%name=Layout,
%plural=Layouts,
%description={Das Layout beschreibt die Gestaltung oder das Aussehen z.B. einer (Internet-) Seite oder einer Programmoberfläche(GUI). Hierzu zählen Formatierungen und Anordnung der Inhalte, wie z.B. Schriftarten, Schriftgrößen, die Seitengröße oder Farben und andere Designelemente.}
%}
%\newglossaryentry{virtuelle Maschine}{
%name=virtuelle Maschine,
%plural=virtuelle Maschinen,
%description={Eine virtuelle Maschine (VM) bildet die Rechnerarchitektur (Hardware) eines realen Rechners in einer abstrahierende Schicht virtuell nach. So wird es möglich, die Hardwareperformance eines realen Rechners auf mehre virtuelle Maschinen aufzuteilen und so einen realen Rechner in mehrere virtuelle, eigenständige und unabhängige virtuelle Rechner aufzuteilen.}
%}
%\newglossaryentry{PDF-Viewer}{
%name=PDF-Viewer,
%description={Ein PDF-Viewer oder auch PDF Reader genannt, ist in Programme zum Betrachten von PDF-Dokumente}
%}
%
%
%\newglossaryentry{Tag}{
%name=Tag,
%plural=Tags,
%description={Ein Tag ist ein Schlüsselwort (Keywort) das eine definierte Aufgabe oder Funktion, im Kontext in dem es genutzt wird, besitzt. }
%}
%\newglossaryentry{Multimedia}{
%name=Multimedia,
%description={Dateien, die gleichzeitig aus unterschiedlichen, i.d.r digitalen Medien zusammenwirken. Z.B. Text, Grafik, Audio, Video oder  Animation.}
%}
%\newglossaryentry{Content stream}{
%name=Content stream,
%plural=Content streams,
%description={Ein Content stream liefert den Inhalt einer PDF Seite. Er enthält eine Sequenz von Anweisungen welche zum einen das Layout einer Seite beschreiben. Zum anderen liefern Content streams die Inhalte der PDF Seiten. Z.B Bilder oder Texte.}
%}
%\newglossaryentry{Library}{
%name=Library,
%description={Bibliothek welche eine Sammlung von Unterprogrammen/-Routinen enthält},
%plural=Libraries
%}
%\newglossaryentry{verkettete Liste}{
%name=verkettete Liste,
%plural=verkettete Listen,
%description={Eine (zyklisch) verkettete Liste ist eine dynamische Datenstruktur zur Speicherung von Objekten. Dabei muss die Anzahl der Objekte im Vorfeld nicht bestimmt werden. Es können beliebig viele Objekte nachträglich in die Liste aufgenommen werden. Durch einen Zeiger wird jeweils auf das nachfolgende Objekt der Liste (bzw. Element oder Knoten)  gezeigt. Oder auf dessen Speicherzellen im Arbeitsspeicher. Anders als (Objekt-) Bäumen sind Listen linear, dass heisst ein Element hat genau einen Nachfolger. Bei einer zyklischen Liste zeigt das letzte Element auf ein vorheriges Element (oftmals das erste Element in der Liste). Bei einer doppelt verketten (zyklischen) Liste existiert ein zweiter Zeiger, welcher auf das vorherige Element einer Liste zeigt. } }
%
%
%
%
\newglossaryentry{Feature}{
name=Feature,
description={Merkmal oder Eigenschaft, wie eine besondere Funktionalität}
}
%
%\newglossaryentry{ExtensionLevel}{
%name=ExtensionLevel,
%description={Erweiterungen oder Änderungen der letzten Revision des PDF Formates. Diese werden als eigene Dokumente veröffentlicht.}
%}
%\newglossaryentry{Acrobat}{
%name=Adobe Acrobat,
%description={Adobe Acrobat ist eine Gruppe von Programmen mit welcher PDF-Datei erstellt, verwalten, kommentiert und verteilt werden können. Dieses kostenpflichtige Programmpaket des Software-Unternehmens Adobe Systems enthält ein Anwendungsprogramm zum Erstellen und Bearbeiten von PDF-Dokumenten.}
%}
%\newglossaryentry{deklariert}{
%name=deklariert,
%plural=deklarieren,
%description={etwas bekannt geben}
%}
%\newglossaryentry{deklaration}{
%name=Deklaration,
%plural=deklarieren,
%description={Bekanntmachung}
%}
%\newglossaryentry{Header}{
%name=Header,
%description={Kopfteil einer Datei, welcher oft Grundlegende Einstellungen definiert.}
%}
%\newglossaryentry{Body}{
%name=Body,
%description={Hauptteil einer Datei.}
%}
%\newglossaryentry{Cross Referenz Tabelle}{
%name=Cross Referenz Tabelle,
%plural=Cross Referenz Tabellen,
%description={Cross Reference Tabellen oder Kreuztabellen, kurz XRef, enthalten Verweise auf definierte Stellen in einer Datei. Dieses ermöglicht einen schnelleren Zugriff auf diese Bereiche.}
%}
%\newglossaryentry{Trailer}{
%name=Trailer,
%description={Schlussteil oder Anhang einer Datei. Beschreibt zusätzlichen Informationen zu einer Datei.}
%}
\newglossaryentry{Root}{
name=Root,
description={Wurzel, Ursprung von dem etwas ausgeht. Im Unix Betriebssystem
    sowohl ein (der) adminstrative Benutzer sowie der Ursprung des
    gesamten Linux Verzeichnisbaums
    (Dateisystem) unter welchem weitere Verzeichnisbäume eingehängt werden.
    Diese können sich auf der selben Festplatte als auch auf einem externen
    Speichermedium befinden.}
}



\newglossaryentry{RootFS}{
name=Root File System,
description={Wurzel (\gls{Root}) des Linux Datei Systems unter welchem weitere
    Verzeichnisbäume (lokal oder extern auf einem anderen Datenträger)
    eingehängt werden. Z.b. \textit{home, bin, tmp, boot, usr, \ldots} }
}

%\newglossaryentry{Integritaet}{
%name=Datenintegrität,
%description={Verhinderung vor unautorisierter Modifikation von Information}
%}
%\newglossaryentry{Vertraulichkeit}{
%name=Vertraulichkeit,
%description={Informationen sind nur für einen beschränkten Empfängerkreis einsehbar. Weitergabe und Veröffentlichung sind dabei i.d.R. nicht erlaubt.}
%}
%\newglossaryentry{Verfuegbarkeit}{
%name=Verfügbarkeit,
%description={Gewährleistung, das ein System oder Service nach definierten Anforderungen verfügbar ist}
%}
%\newglossaryentry{Verbindlichkeit}{
%name=Verbindlichkeit,
%description={Verbindlichkeit oder auch Nichtabstreitbarkeit bedeutet in der Informationstechnik, dass eine (durchgeführte) Sache eindeutig dem Urheber zugeordnet werden kann und er es nicht abstreiten kann.}
%}
%\newglossaryentry{Authentizitaet}{
%name=Authentizität,
%description={Authentizität, beschreibt die Echtheit von etwas. Z.B. einer Datei oder eines Dokumentes. Authentizität beutet sinngemäß: \glqq als Original befunden\grqq\ }
%}
%\newglossaryentry{AES}{
%name=AES,
%description={Der \ac{AES} ist eine symmetrische Blockchiffre mit 128-256 Bit Schlüssellänge}
%}
%\newglossaryentry{RC4}{
%name=RC4,
%description={\ac{RC4} ist eine symmetrische Stromverschlüsselung. Der Klartext wird Bit für Bit per XOR mit der Zufallsfolge verknüpft}
%}
%\newglossaryentry{Flag}{
%name=Flag,
%description={Ein Flag ist ein Statusindikator. Er dient zur Kennzeichnung bestimmter Zustände (z.B. 0 = Licht aus, 1 = Licht an)}
%}
%\newglossaryentry{ASCII}{
%name=ASCII,
%description={\ac{ASCII} ist eine Hexadezimale Zeichencodierung auf ursprünglich 7 Bit Ebene. Dies entspricht 128 druckbare Zeichen die durch einen individuellen Hexadezimalen Wert dargestellt werden können.}
%}
%
%\newglossaryentry{Plugin}{
%name=Plugin,
%plural=pluins,
%description={Ein Plugin (oder Plug-in)  erweitert oder verändert eine bestehende Software. Z.B. um neue oder veränderte Funktionen. Es ist ein optionales zusätzliches Software-Modul für diese Software, dass nicht ohne eine Hauptanwendung ausgeführt werden kann. Teilweise wird statt Plugin auch der Begriff Addon oder Add-on verwendet.}
%}
%
%
%
%\newglossaryentry{Addon}{
%name=Addon,
%plural=Addons,
%description={Ein Addon (oder Add-on)  erweitert (oder verändert) eine bestehende Software. Z.B. um neue oder veränderte Funktionen. Es ist ein optionales zusätzliches Modul dass nicht ohne eine Hauptanwendung genutzt werden kann. Häufig, speziell in der Informatik, wird statt Plugin auch der Begriff Plugin oder Plug-in verwendet.}
%}
%
%\newglossaryentry{Wiki}{
%name=Wiki,
%plural=Wikis,
%description={Ein Wiki ist ein Webseitensystem. Die Inhalte (Texte, Grafiken,.. ) der Webseiten können von andern Benutzern gelesen auch auch direkt online im Webbrowser geändert werden. Ohne eine zusätzliche Software zu benötigen. Es wird häufig genutzt, um Erfahrung und Wissen (gemeinschaftlich) zu sammeln und (verständlich) zu dokumentieren.}
%}
%
%\newglossaryentry{Forum}{
%name=Forum,
%plural=Foren,
%description={Ein Forum ist ein realer oder virtueller (online) Ort, an dem Meinungen zwischen vermieden Personen ausgetauscht werden können, Fragen gestellt und beantwortet werden können oder Themen diskutiert werden können. }
%}
%\newglossaryentry{PostScript}{
%name=PostScript,
%description={PostScript ist eine Programmiersprache, spezialisiert auf Bilder, Texte oder andere Grafische Formen. Diese können z.B. auf einer digitalen Seite dargestellt, oder an einen Drucker weitergeleitet werden. PostScript ist unter \cite{post3} von Adobe spezifiziert. Weitere Informationen sind auch unter \cite{post1} und \cite{post2} zu finden. PostScript ist Grundlage, auf welche das \ac{PDF} aufsetzt.}
%}
%
%\newglossaryentry{Hexadezimal}{
%name=Hexadeimal,
%description={Beim Hexadezimalsystem in einem Stellenwertsystem die Zahlen jeweils zur Basis 16 dargestellt. Dieses ist komfortablere Darstellung als Binärsystems (0 und 1) welches speziell in der eine Rolle spielt. So müssen die Oktette beim Hexadezimalsystem nicht als acht stellige Binärzahlen sonder nur als zweistellige Hexadezimalzahlen  dargestellt werden.},
%plural=Hexadeimaler
%}
%\newglossaryentry{Oktal}{
%name=Oktal,
%description={Das Oktale Zahlensystem wird zur Basis 8 dargestellt. Genau wie die Hexadezimale Schreibweise ist es eine komfortablere Darstellung als Binärsystems (0 und 1). Beim Oktalen Zahlensystem werden Zeichen nicht als acht stellige Binärzahlen sonder als dreistellige Oktalzahlen dargestellt.},
%plural=octalen
%}
%
%\newglossaryentry{Integer}{
%name=Integer,
%description={Eine Ganzzahl. Also eine Zahl ohne Komma und Nachkommastellen.},
%plural=Integers
%}
%\newglossaryentry{Reale Zahl}{
%name=Reale Zahl,
%description={Eine Reahle Zahl ist eine Zahl mit Nachkommastellen},
%plural=Reale Zahlen
%}
%
%\newglossaryentry{Escapezeichen}{
%name=Escapezeichen,
%description={Escapezeichen Leiten in der Informatik eine Sonderfunktion ein. Der Anschließende Buchstabe wird dabei nicht mehr als Buchstabe interpretiert, sonder hat eine definierte Funktionalität. Z.B. $\backslash$n für eine neue Zeile oder $\backslash$r für einen Zeilenrücksprung},
%plural=escaped
%}
%
%
%
%\newglossaryentry{Key-Value-Paar}{
%name=Key-Value-Paar,
%plural=Key-Value-Paare,
%description={Einem Schlüsselwort/Bezeichner/Variable (Key) wird ein Wert (Value) zugeordnet. Es erfolgt also eine Art von Definition, bei der zu einem Schlüssel ein bestimmten Wert festgelegt (definiert) wird}
%}
%
%\newglossaryentry{Abwaertskompatibel}{
%name=Abwärtskompatibel,
%plural=Abwärtskompatibilität,
%description={Ein Objekt (z.B. ein Programm) kann andere Objekte (z.B. eine Datei) richtig verarbeiten, die von einer älteren Version stammen und somit anders zu interpretieren ist oder weniger bzw. andere Funktionalitäten besitzt.}
%}
%
%\newglossaryentry{Aufwaertskompatibel}{
%name=Aufwärtskompatibel,
%plural=Aufwärtskompatibilität,
%description={Ein Objekt (z.B. ein Programm) kann andere Objekte (z.B. eine Datei) richtig verarbeiten, die von einer neueren Version sind, als es selbst. Neuere Versionen können unbekannte Funktionalitäten besitzt oder anders zu interpretieren sein. Dieses macht eine Aufwärtskompatibilität schwierig}
%}
%
%\newglossaryentry{Zeilenruecksprung}{
%name=Zeilenrücksprung,
%description={Dieser Befehl besagt, das der Curser zum Anfang der Zeile Springen soll, bzw. das mit einer neuen Zeile begonnen werden soll. Häufig wird es durch ein $\backslash$r gekennzeichnet.}
%}
%\newglossaryentry{Zeilenumbruch}{
%name=Zeilenumbruch,
%description={Dieser Befehl besagt, das der Curser in der nächsten Zeile beginnen soll. Häufig wird es durch ein $\backslash$n gekennzeichnet. Bei Windowssystem inkludiert ein Zeilenumbruch auch immer einen Zeilenrücksprung durch $\backslash$r. Dieses wird bei Unixsystemen getrennt}
%}
%\newglossaryentry{Encodieren}{
%name=Encodieren,
%description={Ein Datenpaket (wie z.B. einen Text) mit einem Code verschlüsseln},
%plural=encodieren
%}
%\newglossaryentry{PDF Processor}{
%name=PDF Processor,
%plural=PDF Prozessoren,
%description={Ein PDF Processor ist jede Art von Software oder Hardware, welche PDF Dateien schreiben, lesen aktualisieren (updaten) oder anderweitig verarbeitet. Grundlegend wird zwischen \gls{PDF Writer} (PDF Software oder Hardware zum Schreiben von PDF Dateien) und PDF Readern (PDF Software oder Hardware zum Anzeigen von PDF Dateien) unterschieden.  }
%}
%\newglossaryentry{escape}{
%name=escape,
%plural=escaped,
%description={Unter escape (Escapen) versteht man in der Informationstechnik den Vorgang, eine Sonderfunktion, z.B. eines Zeichens, zu ignorieren und das Zeichen als normales Textzeichen darzustellen, anstatt die Sonderfunktion auszuführen. Hierzu wird wiederum ein Zeichen mit Sonderfunktion benötigt, welche als Escapezeichen bezeichnet wird. }
%}
%\newglossaryentry{case-sensitive}{
%name=case-sensitve,
%description={Unter case-sensitve versteht man, das die Groß und Kleinschreibung berücksichtigt wird \glqq Hallo\grqq\ entspricht somit nicht! dem Wort \glqq hallo\grqq\. Es sind 2 unterschiedliche Wörter. }
%}
%
%\newglossaryentry{PDF Writer}{
%name=PDF Writer,
%description={Ein PDF Writer ist eine Software oder Hardware, mit welcher PDF Dateien erstellt oder erzeugt (z.B. ein PDF Printer) werden können }
%}
%\newglossaryentry{PDF Reader}{
%name=PDF Reader,
%description={Ein PDF Reader (oder auch PDF Viewer) Software oder Hardware zum Anzeigen und Verarbeiten (\gls{PDF Converter}) von PDF Dateien. Abhängig vom Funktionsumfang des PDF Readers können mit diesem auch Formulare ausgefüllt, Inhalte extrahiert, Multimediainhalte wieder gegen werden oder Eingebettete Inhalte und Funktionen geöffnet oder Ausgeführt werden, wie das öffnen einer Internetseite oder eines Dokumentes oder das Ausführen von Programmen}
%}
%\newglossaryentry{PDF Converter}{
%name=PDF Converter,
%description={Software oder Hardware, welche PDF Dateien in andere Dateiformate umwandelt. Oder andere Dateiformate in PDF Dateien umwandelt.}
%}
%\newglossaryentry{PDF Printer}{
%name=PDF Printer,
%description={Ein virtueller Drucker (oder vielmer Druckertreiber), welcher empfangene Daten in PDF Syntax umwandelt und das Ergebniss als PDF Dateien abspeichert. Es wird somit ein PDF Dokument erzeugt, anstatt Inhalte Physikalisch auszudrucken.}
%}
%
%\newglossaryentry{PDF Drucker}{
%name=PDF Drucker,
%description={Ein virtueller Drucker (oder vielmer Druckertreiber), welcher empfangene Daten in PDF Syntax umwandelt und das Ergebniss als PDF Dateien abspeichert. Es wird somit ein PDF Dokument erzeugt, anstatt Inhalte Physikalisch auszudrucken.}
%}

\newglossaryentry{Meta}{
name=Metadaten,
description={Metadaten oder Metainformationen sind Informationen über andere Daten}
}

\newglossaryentry{Metadatendatei}{
name=Metadatendatei,
description={Eine Datei, welche \glspl{Meta} (Informationen zu anderen Daten) über ndere Daten enthält.}
}
%\newglossaryentry{End Of Line}{
%name=End Of Line,
%description={Symbolisiert das \glqq Ende einer Zeile\grqq\.}
%}
%\newglossaryentry{Token}{
%name=Token,
%description={PDF Dateien bestehen aus einer Sequenz von 8 Bit Binärbytes. Sie werden zu Gruppen zusammen gefasst die Schlüsselwörter, Strings, Numbers, usw repräsentieren. Eine Gruppe von Byte wird in PDF als Tokens definiert \cite[S. 48]{pdf}}
%}
%%\newglossaryentry{case-sensitive}{
%%name=case-sensitive,
%%description={case-sensitive bedeutet, das zwischen Klein und Großbuchstaben unterschieden wird. "\glqq Ein  Beispiel\grqq\ ist also \textbf{nicht!} gleich \glqq ein beispiel\grqq\ .}
%%}
%
%\newglossaryentry{Font}{
%name=Font,
%plural=Fonts,
%description={Als Font wird der Informationstechnik jede Schriftart bezeichnet, die auf einem Computer oder Peripheriegerät vorhandene ist.}
%}
%\newglossaryentry{decodieren}{
%name=decodieren,
%plural=decodiert,
%description={Eine Datei oder Nachricht wieder Entschlüsseln und ins ursprüngliche Format zurück übersetzten.}
%}
%\newglossaryentry{Offset}{
%name=Offset,
%description={Eine bestimmte Speicheradresse oder eine Anzahl Bits, Bytes, ..., die i.d.R vom Beginn einer Datei berechnet berechnet wird. }
%}
%\newglossaryentry{Hybride PDF Datei}{
%name=Hybride PDF Datei,
%plural=hybriden PDF Dateien,
%description={PDFs, die zusätzlich zu Objektstreams und einem \gls{XRef} Stream (verfügbar ab PDF Version 1.5), auch normale Objekte und eine normale \gls{XRef} Tabelle enthalten (bis PDF Version 1.4 die einzige Form). Siehe hierzu das Kabitel \glqq Compatibility with Applications That Do Not Support PDF 1.5\grqq\ der Adobe Referenz unter \cite[S. 109 ff]{pdf2}}
%}
%
%\newglossaryentry{Update}{
%name=Update,
%plural=Updates,
%description={Ein Update ist eine Aktualliserung. Etwas wird auf eine aktuellen Stand oder eine aktuelle Version gebracht. }
%}
%
%\newglossaryentry{inkrementell}{
%name=inkrementell,
%plural=inkrementelle,
%description={schrittweise erfolgend, aufeinander aufbauend. \cite[Schlagwort: inkrementell]{duden} }
%}
%
%
%\newglossaryentry{EOL}{
%name=End of Line,
%description={Mit \ac{EOL} wird das Ende einer Zeile beschrieben. Der Marker beschreibt in der Informatik häufig die Kombination der Befehle für eine neue Zeile (meist \bb n) und einen Zeilenrücksprung (meist \bb r)}
%}
%\newglossaryentry{EOF}{
%name=End of File,
%description={Mit \ac{EOF} wird das Ende einer Datei beschrieben. Wie dieser Marker aussieht ist abhängig vom Programm, welches eine solche Markierung zum Lesen einer Datei benötigt.}
%}
%\newglossaryentry{Hash}{
%name=Hash,
%plural=Hases,
%description={Ein Hash ist eine bezeichnet eine Verkürzte Form von etwas. Durch eine Hashfunktion (der Vorgang um einen Hash zu erstellen) wird eine Große Eingabe (z.B. eine Datei oder ein Satz) in eine kleine Ausgabemenge umgewandelt(abgebildet). Es existieren verschiedene Verfahren, um einen Hash (oder Hashwert) zu erzeugen. Die bekanntesten sind \ac{MD5} und \ac{SHA} in Version 2 und 3. Ein Hashwert wird häufig als Prüfsumme verwendet und wird in diesem Fall auch als Fingerprint bezeichnet.}
%}

\newglossaryentry{Tool}{
name=Tool, description={Ein Tool (Informationstechnik, zu Deutsch
    \textit{Werkzeug})  beschreib beschreibt ein Stück Software das
    Hilfsaufgaben ausführt um eingesetzt bei anderen Aufgaben zu unterstützen }}


 % ...oder alternativ input statt loadglsentries

%    \usepackage[acronym]{glossaries}



\begin{document}

%Definitionen neuer Befehle
\newcommand{\bb}{$\backslash$} %Backshlash


\setcounter{secnumdepth}{3}

% Titelblatt
\include{title}
\cleardoubleemptypage

\frontmatter


% Abstract
\include{content/abstract}
\cleardoubleemptypage

% Inhaltsverzeichnis
\tableofcontents
\addcontentsline{toc}{chapter}{Inhaltsverzeichnis}
\cleardoubleemptypage

% Abbildungsverzeichnis einbinden und ins Inhaltsverzeichnis
% WORKAROUND: tocloft und KOMA funktionieren zusammen nicht
% korrekt\phantomsection
\addcontentsline{toc}{chapter}{\listfigurename}
\listoffigures
\cleardoubleemptypage

% Tabellenverzeichnis einbinden und ins Inhaltsverzeichnis
% WORKAROUND: tocloft und KOMA funktionieren zusammen nicht
% korrekt\phantomsection
%\phantomsection
%\addcontentsline{toc}{chapter}{\listtablename}
%\listoftables
%\cleardoubleemptypa\usepackage[printonlyused]{acronym}ge

% Abkürzungsverzeichnis
\chapter{Abkürzungsverzeichnis}
\begin{acronym}
    \acro{SDK}{Software Development Kit}
    \acro{DTB}{Device Tree Blob}
    \acro{BSP}{Board Support Packed}
    \acro{TFTP}{Trivial File Transfer Protocol}
    \acro{NFS}{Network File System}
    \acro{FTP}{File Transport Protocol}
    \acro{WYSIWYG}{What you see is what you get}

    \acro{PDF}{Portable Dokument Format}
    \acro{HFU}{Hochschule Furtwangen}
    \acro{ISO}{International Organization for Standardization}
    \acro{AES}{Advanced Encryption Standard}
    \acro{RC4}{Arcfour}
    \acro{ASCII}{American Standard Code for Information Interchange}
    \acro{XRef}{Cross Referenz}
    \acro{HTTP}{Hypertext Transfer Protocol}
    \acro{OLE}{Object Linking and Embedding}
    \acro{EOL}{End of Line}
    \acro{EOF}{End of File}
    \acro{MD5}{Message-Digest Algorithm Version 5}
    \acro{SHA}{Secure Hash Algorithm}
    \acro{ES}{Escaptes Sonderzeichen}
    \acro{SZ}{Sonderzeichen}
    \acro{FDF}{Forms Data Format}
    \acro{PJDF}{Portable Job Ticket Format}
    \acro{HTTP}{Hypertext Transfer Protocol}
    \acro{CMYK}{Cyan, Magenta, Yellow und der Schwarzanteil Kay}
    \acro{RGB}{Rot, Grün und Blau}
    \acro{VDP}{Variable Data Printing}
    \acro{PAdES}{PDF Advanced Electronic Signatures}
    \acro{SP}{Space Character}
    \acro{HT}{Horizontal Tabulator}
    \acro{CR}{Carriage return}
    \acro{LF}{Line feed}
    \acro{FF}{Form feed}
    \acro{NUL}{Null character}
    \acro{OID}{Object Identifier}
    \acro{ETSI}{European Telecommunications Standards Institute}
    \acro{VM}{virtuelle Maschine}
\end{acronym}



% %Gossary
% %https://en.wikibooks.org/wiki/LaTeX/Glossary
%% % % % % % % % % % % % % % % % %Glossar
%\usepackage{ngerman}
%\usepackage[ngerman]{babel}

\newglossaryentry{NFS}{
name=NFS,
description={ Das \acf{NFS}, (auch Network File Service) ist ein
    Netzwerkprotokoll, das den Zugriff auf Dateien über ein Netzwerk ermöglicht.
    Dabei werden die Dateien nicht wie z. B. bei \ac{FTP} oder \ac{TFTP}
    übertragen, sondern die Benutzer können auf Dateien, die sich auf einem
    entfernten Rechner befinden, so zugreifen, als ob sie auf ihrer lokalen
    Festplatte abgespeichert wären. Unter Unix Betriebssystemen lassen sich
    diese Netzwerkfreigaben direkt in das Dateisystem einhängen entsprechend wie
z.B. Festplatten. }
}


\newglossaryentry{TFTP}{
name=TFTP,
description={Das \acf{TFTP} ist ein im
    Fumktionsumfang stark vereinfachtes \acl{FTP} und reduziertes Protokoll, das
lediglich das senden und das empfangen von Dateien ermöglicht.}
}

\newglossaryentry{GIT}{
name=GIT,
description={ Git ist eine freie Software zur verteilten Versionsverwaltung von
    Dateien, die durch Linus Torvalds initiiert wurde. Es arbeiten im gegensatz
    zu anderen Versionsverwaltungen mit lokalen als auch externen Reposetories.
}
}

\newglossaryentry{Dockerfile}{
name=Dockerfile,
description={Ein Dockerfile beschreibt wie ein \gls{Docker} Container (Image)
    gebaut werden soll, genauer welche Tools und Konfigurationen der
    Container enthalten soll.}
}


\newglossaryentry{Yocto}{
name=Yocto,
description={
Das Yocto Project bezeichnet eine Community Gruppe, welche
sich zur Aufgabe gemacht hat, dass erstellen von Linux Distributionen für
eingebettet System zu vereinfachen. Die Yocto community stellt Meta-Daten (Recipes) bereit, um ein minimalistisches
Betriebssystem mit grundlegenden  Linux \glspl{Tool} unter der Virtualisierungsumgebung
\textit{\gls{QEMU}} für verschiedene Architekturen starten zu können.
Das Minimalisten Betriebssystem der Yocto Community wird \textit{\gls{Poky}}
genannt. Zudem Pflegt es zusammen mit der \gls{OpenEmbedded} community das Buildsystem
\textit{bitbake}.  }
}


\newglossaryentry{QEMU}{
name=QEMU,
description={QEMU ist (von englisch „Quick Emulator“) ist eine freie
    Virtualisierungssoftware, die die gesamte Hardware mitsamt
    Prozessorinstruktionen einer einer Zielhardware emuliert. Dabei ist es sehr
Recourcenschonend und performant.}
}

\newglossaryentry{Wrapper}{
name=Wrapper,
description={ Ein Wrapper ist ein Sofwareinterface das Interaktion mit einem
    anderen Softwarepaket kapselt und i.d.R. nach außen hin vereinfacht.
}
}



\newglossaryentry{OpenEmbedded}{
name=OpenEmbedded,
description={ Eine Community Gruppe welche eine Softwareware Buildumgebung
namens Bitbake pflegt und Metatdaten für dieses Buildsystem bereitstellt,
welche von dieser Buildumgebung verwendet wird um Softwarepakete zu übersetzen}
}

\newglossaryentry{Recipes}{
name=Recipies,
description={Rezepte oder Anleitungen}
}


\newglossaryentry{Metadaten}{
name=Metadaten,
description={Metadaten oder Metainformationen sind Informationen über andere
Daten}
}

\newglossaryentry{Docker}{
name=Docker,
description={Docker ist eine Software zur virtualisierung von einzelnen
    Anwendungen}
}

\newglossaryentry{SDK}{
    name=SDK,
    description={Ein \ac{SDK} ist eine gekapselte bzw.
vordefinierte Umgebung zur Entwicklung von Software\-komponenten. Die Umgebung
stellt (vorkonfigurierte) Sammlungen von ausgewählten Programmier\-werkzeugen
oder auch Liberias bereit, die zur Entwicklung für eine Zielplattform benötigt
werden.}
}

\newglossaryentry{rootfs}{
    name=rootfs,
    description={Unter Linux wird das urspüngliche  \textit{rootfs} als der Ort
        bezeichnet, von  dem \textbf{ausgehend} alle weiteren Verzeichnissbäume
        eingehängt sind/werden. Beispielweise liegen direkt unterhalb des rootfs
        die Ordner /home, /boot /root oder /bin.}
}

\newglossaryentry{Workflow}{
    name=Workflow,
    description={Ein Wokflow ist ein Arbeitsablauf und beschreibt Schritte in
        ihrer Reihenfolge die nötig sind, um eine Aufgabe, Arbeitspaket oder
        etwa eine Anweisung zu erfüllen. Dabei sind die Arbeitsschritte häufig
        wiederkehrend in anderen Arbeitsabläufen.
        } }

\newglossaryentry{dummy}{
    name=dummy,
    description=dummy{
        } }

\newglossaryentry{Poky}{
    name=Poky,
    description={Minimalistisches Betriebssystem der \gls{Yocto} Community das
        mitsamt seiner Tools in einer \gls{QEMU} virtuellen Maschine lauffähig
        ist.
        } }

\newglossaryentry{Distribution}{
    name=Distribution,
    plural=Distributionen,
    description={Als Distribution bezeichnet man eine Zusammenstellung von
        (Software-) Paketen,Versionen,  Konfigurationen, Einstellungen, usw.
        die als Gesamtpaket veröffentlicht sind oder werden und in sich ohne
        weiteres Zutun eine definierte Aufgabe erfüllen. Beispielsweise ist
        ein Betriebssystem eine solche Zusammenstellung das ohne weiters Zutun
        für einen Satz von Anwendungsfällen genutzt werden kann.
        } }


\newacronym{DTB}{DTB}{Device Tree Blob} \newacronym{BSP}{BSP}{Board Support
    Packed}


%
%
%\newglossaryentry{Mircosoft Office}{
%name=Mircosoft Office,
%description={Ein Softwarepaket, welches verschiedene Programme für unterschiedliche Aufgaben enthält. Weiteres ist unter
%\cite{ms1}}
%}
%
%\newglossaryentry{Portables Job Definition Format}{
%name=Portables Job Definition Format,
%description={
%\ac{PJDF} ist ein von Adobe entwickeltes Format zur Speicherung technischer Produktions- und Auftragsdaten auf Basis der PDF Format Syntax. \cite{pjdf1}}
%}
%
%\newglossaryentry{32-Bit Architektur}{
%name=32-Bit Architektur,
%plural=32-Bit Architekturen,
%description={32-Bit kennzeichnet eine PC bzw. Prozessor oder Platinen Architektur. Sie kennzeichnet die Breite des Adressbusses und gibt somit vor, wie viel Speicher (adressierbare Blöcke oder Einheiten mit jeweils einem Byte) eine Architektur ansprechen und verarbeiten kann. Mit einem 32-Bit-Adressbus lassen sich maximal 2e32 Byte adressieren. Dies sind 232 Speicherstellen mit jeweils einem Byte. Oder umgerechnet 4 GiB. Die Architekturform hat vor allem Einfluss auf die Prozessor- und Busarchitektur. Diese hängen unmittelbar zusammen und sind in Bezug auf Ihre Geschwindigkeit abhängig von der Architekturform (in diesem Fall beschränkt auf 32 Bit). Hieraus ergibt sich auch der Begriff \glqq 32-Bit Betriebssystem\grqq\, welches auf die Prozessor- und Busarchitektur aufsetzt. Einem 32-Bit Betriebssystem ist es somit auch nur möglich, Max. 4 GB (bzw. GiB) Arbeitsspeicher anzusprechen und zu verarbeiten (teilweise sogar weniger).  }}
%
%\newglossaryentry{32-Bit Betriebssystem}{
%name=32-Bit Betriebssystem,
%plural=32-Bit Betriebssysteme,
%description={Siehe \gls{32-Bit Architektur}}
%}
%
%\newglossaryentry{Datentyp}{
%name=Datentyp,
%plural=Datentypen,
%description={Datentyp beschreibt den Wertebereich von \glspl{Variable}. Sie werden daher auch als Werteart oder Datenart beschrieben. Sie sind gekennzeichnet durch einen Wertebereich, einen Name, einen Geltungsbereich, einer Konstruktionsregel, und einer gültigen (zulässige) Menge von Operationen über diese Wertemenge. Es wird zwischen numerische, boolesche, alphanumerische, abstrakte, elementare, zusammengesetzte und benutzerdefinierte Datentypen unterschieden. }
%}
%
%
%\newglossaryentry{Variable}{
%name=Variable,
%plural=Variablen,
%description={Eine Variable ist in der Informatik ein Platzhalter. Er ist gekennzeichnet durch einen Namen und häufig durch einen Datentyp. Variablen werden (temporäre) Werte zugeordnet. Diese können verändert, überschrieben oder gelöscht werden. Sie dienen häufig als (Zwischen-) Speicher, z.B. für eine Berechnung. }
%}
%
%\newglossaryentry{Konversationen}{
%name=Konversationen,
%description={Einigung, Normung oder Übereinkunft z.B. über einen Zeichensatzes. Zu dieser Einigung ist man gemeinsam in einem Gespräch (Konversation) gekommen. Diese Einigung ist häufig zu einer De-facto- oder Quasi-Vorgabe geworden.}
%}
%\newglossaryentry{Forms Data Format}{
%name=Forms Data Format,
%description={\ac{FDF} ist ein Format zur (Serverseitigen-) Verarbeitung der Formulardaten innerhalb eines PDF Dokumenten \cite{fdf1} \cite{fdf3} \cite{fdf2}}
%}
%\newglossaryentry{String}{
%name=String,
%plural=Strings,
%description={Ein String bezeichnet eine Zeichenkette in der Informatik. Sie repräsentiert z.B. ein Wort oder einen Satz.}
%}
%\newglossaryentry{Hypertext Transfer Protocol}{
%name=Hypertext Transfer Protocol,
%description={Das Hypertext Transfer Protocol (HTTP) ist ein Protokoll zur Übertragung von Daten auf der Anwendungsschicht in Rechnernetzen. Meistens wird es genutzt Hypertext-Dokumente (Webseiten) aus dem World Wide Web (WWW) in einen Webbrowser zu laden. Das Protokoll ist Zustandslos}
%}
%\newglossaryentry{Backdoor}{
%name=Backdoor,
%description={Hintertür für Angreifer, damit diese sich zu dem Rechner verbinden und diesen eindringen können.}
%}
%\newglossaryentry{Owner}{
%name=Owner,
%description={Als \glqq Owner\grqq\ wird der Besitzer von etwas bezeichnet. Z.b. der Besitzer (Owner) einer Datei oder eines Dokumentes. Er ist oft gleichzusetzen mit einem Administrator. Der Owner besitzt i.d.R. die vollen Zugriffsrechte, inklusive Rechte zum Verändern der Inhalte oder Löschen der kompletten Datei.}
%}
%\newglossaryentry{Pubic-Private Key Verfahren}{
%name=Pubic-Private Key Verfahren,
%description={Ein Public-Privat-Key Verschlüsselungsverfahren (kurz Public-Key Verfahren) ein kryptographisches Schutz Verfahren. Bei diesem wird mit einem veröffentlichten (öffentlicher) Schlüssel eine Klartextdatei in eine verschlüsselte Datei umgewandelt. Die Datei kann nur wieder durch einen zweiten, geheimen Schlüssel in Klartextform umgewandelt werden. Dieses wird asymmetrisches Verschlüsselungsverfahren bezeichnet, da zwei unterschiedliche Schlüssel (ein öffentlicher und ein geheimer) benötigt werden um ein Dokument zu verschlüsseln und weder zu entschlüsseln.}
%}
%
%
%%\newglossaryentry{XRef Sektion}{
%%name=XRef Sektion,
%%plural=XRef Sektionen,
%%description={Eine Cross Referenz Sektion (XRef Sektion) beschreibt den Bereich einer Cross Referenz Tabelle oder des Cross Referenz Stream Objektes in einem PDF Dokument}
%%}
%
%\newglossaryentry{User}{
%name=User,
%description={Als \glqq User\grqq\ wird ein normaler Benutzer bezeichnet. Er besitzt häufig weniger, bzw. eingeschränkte (Zugriffs-) Rechte, etwa auf eine Datei oder Ordner. (Im Gegensatz zu einem Owner oder Administrator.) Häufig darf er etwa Dateien nur lesen oder eingeschränkt bearbeiten und verwenden.}
%}
%\newglossaryentry{Adobe Reader}{
%name=Adobe Reader,
%description={Software zu Anzeigen von PDF Dateien. Es stammt vom Hersteller Adobe, welcher auch das \ac{PDF} Format entworfen hat.}
%}
%\newglossaryentry{Layout}{
%name=Layout,
%plural=Layouts,
%description={Das Layout beschreibt die Gestaltung oder das Aussehen z.B. einer (Internet-) Seite oder einer Programmoberfläche(GUI). Hierzu zählen Formatierungen und Anordnung der Inhalte, wie z.B. Schriftarten, Schriftgrößen, die Seitengröße oder Farben und andere Designelemente.}
%}
%\newglossaryentry{virtuelle Maschine}{
%name=virtuelle Maschine,
%plural=virtuelle Maschinen,
%description={Eine virtuelle Maschine (VM) bildet die Rechnerarchitektur (Hardware) eines realen Rechners in einer abstrahierende Schicht virtuell nach. So wird es möglich, die Hardwareperformance eines realen Rechners auf mehre virtuelle Maschinen aufzuteilen und so einen realen Rechner in mehrere virtuelle, eigenständige und unabhängige virtuelle Rechner aufzuteilen.}
%}
%\newglossaryentry{PDF-Viewer}{
%name=PDF-Viewer,
%description={Ein PDF-Viewer oder auch PDF Reader genannt, ist in Programme zum Betrachten von PDF-Dokumente}
%}
%
%
%\newglossaryentry{Tag}{
%name=Tag,
%plural=Tags,
%description={Ein Tag ist ein Schlüsselwort (Keywort) das eine definierte Aufgabe oder Funktion, im Kontext in dem es genutzt wird, besitzt. }
%}
%\newglossaryentry{Multimedia}{
%name=Multimedia,
%description={Dateien, die gleichzeitig aus unterschiedlichen, i.d.r digitalen Medien zusammenwirken. Z.B. Text, Grafik, Audio, Video oder  Animation.}
%}
%\newglossaryentry{Content stream}{
%name=Content stream,
%plural=Content streams,
%description={Ein Content stream liefert den Inhalt einer PDF Seite. Er enthält eine Sequenz von Anweisungen welche zum einen das Layout einer Seite beschreiben. Zum anderen liefern Content streams die Inhalte der PDF Seiten. Z.B Bilder oder Texte.}
%}
%\newglossaryentry{Library}{
%name=Library,
%description={Bibliothek welche eine Sammlung von Unterprogrammen/-Routinen enthält},
%plural=Libraries
%}
%\newglossaryentry{verkettete Liste}{
%name=verkettete Liste,
%plural=verkettete Listen,
%description={Eine (zyklisch) verkettete Liste ist eine dynamische Datenstruktur zur Speicherung von Objekten. Dabei muss die Anzahl der Objekte im Vorfeld nicht bestimmt werden. Es können beliebig viele Objekte nachträglich in die Liste aufgenommen werden. Durch einen Zeiger wird jeweils auf das nachfolgende Objekt der Liste (bzw. Element oder Knoten)  gezeigt. Oder auf dessen Speicherzellen im Arbeitsspeicher. Anders als (Objekt-) Bäumen sind Listen linear, dass heisst ein Element hat genau einen Nachfolger. Bei einer zyklischen Liste zeigt das letzte Element auf ein vorheriges Element (oftmals das erste Element in der Liste). Bei einer doppelt verketten (zyklischen) Liste existiert ein zweiter Zeiger, welcher auf das vorherige Element einer Liste zeigt. } }
%
%
%
%
\newglossaryentry{Feature}{
name=Feature,
description={Merkmal oder Eigenschaft, wie eine besondere Funktionalität}
}
%
%\newglossaryentry{ExtensionLevel}{
%name=ExtensionLevel,
%description={Erweiterungen oder Änderungen der letzten Revision des PDF Formates. Diese werden als eigene Dokumente veröffentlicht.}
%}
%\newglossaryentry{Acrobat}{
%name=Adobe Acrobat,
%description={Adobe Acrobat ist eine Gruppe von Programmen mit welcher PDF-Datei erstellt, verwalten, kommentiert und verteilt werden können. Dieses kostenpflichtige Programmpaket des Software-Unternehmens Adobe Systems enthält ein Anwendungsprogramm zum Erstellen und Bearbeiten von PDF-Dokumenten.}
%}
%\newglossaryentry{deklariert}{
%name=deklariert,
%plural=deklarieren,
%description={etwas bekannt geben}
%}
%\newglossaryentry{deklaration}{
%name=Deklaration,
%plural=deklarieren,
%description={Bekanntmachung}
%}
%\newglossaryentry{Header}{
%name=Header,
%description={Kopfteil einer Datei, welcher oft Grundlegende Einstellungen definiert.}
%}
%\newglossaryentry{Body}{
%name=Body,
%description={Hauptteil einer Datei.}
%}
%\newglossaryentry{Cross Referenz Tabelle}{
%name=Cross Referenz Tabelle,
%plural=Cross Referenz Tabellen,
%description={Cross Reference Tabellen oder Kreuztabellen, kurz XRef, enthalten Verweise auf definierte Stellen in einer Datei. Dieses ermöglicht einen schnelleren Zugriff auf diese Bereiche.}
%}
%\newglossaryentry{Trailer}{
%name=Trailer,
%description={Schlussteil oder Anhang einer Datei. Beschreibt zusätzlichen Informationen zu einer Datei.}
%}
\newglossaryentry{Root}{
name=Root,
description={Wurzel, Ursprung von dem etwas ausgeht. Im Unix Betriebssystem
    sowohl ein (der) adminstrative Benutzer sowie der Ursprung des
    gesamten Linux Verzeichnisbaums
    (Dateisystem) unter welchem weitere Verzeichnisbäume eingehängt werden.
    Diese können sich auf der selben Festplatte als auch auf einem externen
    Speichermedium befinden.}
}



\newglossaryentry{RootFS}{
name=Root File System,
description={Wurzel (\gls{Root}) des Linux Datei Systems unter welchem weitere
    Verzeichnisbäume (lokal oder extern auf einem anderen Datenträger)
    eingehängt werden. Z.b. \textit{home, bin, tmp, boot, usr, \ldots} }
}

%\newglossaryentry{Integritaet}{
%name=Datenintegrität,
%description={Verhinderung vor unautorisierter Modifikation von Information}
%}
%\newglossaryentry{Vertraulichkeit}{
%name=Vertraulichkeit,
%description={Informationen sind nur für einen beschränkten Empfängerkreis einsehbar. Weitergabe und Veröffentlichung sind dabei i.d.R. nicht erlaubt.}
%}
%\newglossaryentry{Verfuegbarkeit}{
%name=Verfügbarkeit,
%description={Gewährleistung, das ein System oder Service nach definierten Anforderungen verfügbar ist}
%}
%\newglossaryentry{Verbindlichkeit}{
%name=Verbindlichkeit,
%description={Verbindlichkeit oder auch Nichtabstreitbarkeit bedeutet in der Informationstechnik, dass eine (durchgeführte) Sache eindeutig dem Urheber zugeordnet werden kann und er es nicht abstreiten kann.}
%}
%\newglossaryentry{Authentizitaet}{
%name=Authentizität,
%description={Authentizität, beschreibt die Echtheit von etwas. Z.B. einer Datei oder eines Dokumentes. Authentizität beutet sinngemäß: \glqq als Original befunden\grqq\ }
%}
%\newglossaryentry{AES}{
%name=AES,
%description={Der \ac{AES} ist eine symmetrische Blockchiffre mit 128-256 Bit Schlüssellänge}
%}
%\newglossaryentry{RC4}{
%name=RC4,
%description={\ac{RC4} ist eine symmetrische Stromverschlüsselung. Der Klartext wird Bit für Bit per XOR mit der Zufallsfolge verknüpft}
%}
%\newglossaryentry{Flag}{
%name=Flag,
%description={Ein Flag ist ein Statusindikator. Er dient zur Kennzeichnung bestimmter Zustände (z.B. 0 = Licht aus, 1 = Licht an)}
%}
%\newglossaryentry{ASCII}{
%name=ASCII,
%description={\ac{ASCII} ist eine Hexadezimale Zeichencodierung auf ursprünglich 7 Bit Ebene. Dies entspricht 128 druckbare Zeichen die durch einen individuellen Hexadezimalen Wert dargestellt werden können.}
%}
%
%\newglossaryentry{Plugin}{
%name=Plugin,
%plural=pluins,
%description={Ein Plugin (oder Plug-in)  erweitert oder verändert eine bestehende Software. Z.B. um neue oder veränderte Funktionen. Es ist ein optionales zusätzliches Software-Modul für diese Software, dass nicht ohne eine Hauptanwendung ausgeführt werden kann. Teilweise wird statt Plugin auch der Begriff Addon oder Add-on verwendet.}
%}
%
%
%
%\newglossaryentry{Addon}{
%name=Addon,
%plural=Addons,
%description={Ein Addon (oder Add-on)  erweitert (oder verändert) eine bestehende Software. Z.B. um neue oder veränderte Funktionen. Es ist ein optionales zusätzliches Modul dass nicht ohne eine Hauptanwendung genutzt werden kann. Häufig, speziell in der Informatik, wird statt Plugin auch der Begriff Plugin oder Plug-in verwendet.}
%}
%
%\newglossaryentry{Wiki}{
%name=Wiki,
%plural=Wikis,
%description={Ein Wiki ist ein Webseitensystem. Die Inhalte (Texte, Grafiken,.. ) der Webseiten können von andern Benutzern gelesen auch auch direkt online im Webbrowser geändert werden. Ohne eine zusätzliche Software zu benötigen. Es wird häufig genutzt, um Erfahrung und Wissen (gemeinschaftlich) zu sammeln und (verständlich) zu dokumentieren.}
%}
%
%\newglossaryentry{Forum}{
%name=Forum,
%plural=Foren,
%description={Ein Forum ist ein realer oder virtueller (online) Ort, an dem Meinungen zwischen vermieden Personen ausgetauscht werden können, Fragen gestellt und beantwortet werden können oder Themen diskutiert werden können. }
%}
%\newglossaryentry{PostScript}{
%name=PostScript,
%description={PostScript ist eine Programmiersprache, spezialisiert auf Bilder, Texte oder andere Grafische Formen. Diese können z.B. auf einer digitalen Seite dargestellt, oder an einen Drucker weitergeleitet werden. PostScript ist unter \cite{post3} von Adobe spezifiziert. Weitere Informationen sind auch unter \cite{post1} und \cite{post2} zu finden. PostScript ist Grundlage, auf welche das \ac{PDF} aufsetzt.}
%}
%
%\newglossaryentry{Hexadezimal}{
%name=Hexadeimal,
%description={Beim Hexadezimalsystem in einem Stellenwertsystem die Zahlen jeweils zur Basis 16 dargestellt. Dieses ist komfortablere Darstellung als Binärsystems (0 und 1) welches speziell in der eine Rolle spielt. So müssen die Oktette beim Hexadezimalsystem nicht als acht stellige Binärzahlen sonder nur als zweistellige Hexadezimalzahlen  dargestellt werden.},
%plural=Hexadeimaler
%}
%\newglossaryentry{Oktal}{
%name=Oktal,
%description={Das Oktale Zahlensystem wird zur Basis 8 dargestellt. Genau wie die Hexadezimale Schreibweise ist es eine komfortablere Darstellung als Binärsystems (0 und 1). Beim Oktalen Zahlensystem werden Zeichen nicht als acht stellige Binärzahlen sonder als dreistellige Oktalzahlen dargestellt.},
%plural=octalen
%}
%
%\newglossaryentry{Integer}{
%name=Integer,
%description={Eine Ganzzahl. Also eine Zahl ohne Komma und Nachkommastellen.},
%plural=Integers
%}
%\newglossaryentry{Reale Zahl}{
%name=Reale Zahl,
%description={Eine Reahle Zahl ist eine Zahl mit Nachkommastellen},
%plural=Reale Zahlen
%}
%
%\newglossaryentry{Escapezeichen}{
%name=Escapezeichen,
%description={Escapezeichen Leiten in der Informatik eine Sonderfunktion ein. Der Anschließende Buchstabe wird dabei nicht mehr als Buchstabe interpretiert, sonder hat eine definierte Funktionalität. Z.B. $\backslash$n für eine neue Zeile oder $\backslash$r für einen Zeilenrücksprung},
%plural=escaped
%}
%
%
%
%\newglossaryentry{Key-Value-Paar}{
%name=Key-Value-Paar,
%plural=Key-Value-Paare,
%description={Einem Schlüsselwort/Bezeichner/Variable (Key) wird ein Wert (Value) zugeordnet. Es erfolgt also eine Art von Definition, bei der zu einem Schlüssel ein bestimmten Wert festgelegt (definiert) wird}
%}
%
%\newglossaryentry{Abwaertskompatibel}{
%name=Abwärtskompatibel,
%plural=Abwärtskompatibilität,
%description={Ein Objekt (z.B. ein Programm) kann andere Objekte (z.B. eine Datei) richtig verarbeiten, die von einer älteren Version stammen und somit anders zu interpretieren ist oder weniger bzw. andere Funktionalitäten besitzt.}
%}
%
%\newglossaryentry{Aufwaertskompatibel}{
%name=Aufwärtskompatibel,
%plural=Aufwärtskompatibilität,
%description={Ein Objekt (z.B. ein Programm) kann andere Objekte (z.B. eine Datei) richtig verarbeiten, die von einer neueren Version sind, als es selbst. Neuere Versionen können unbekannte Funktionalitäten besitzt oder anders zu interpretieren sein. Dieses macht eine Aufwärtskompatibilität schwierig}
%}
%
%\newglossaryentry{Zeilenruecksprung}{
%name=Zeilenrücksprung,
%description={Dieser Befehl besagt, das der Curser zum Anfang der Zeile Springen soll, bzw. das mit einer neuen Zeile begonnen werden soll. Häufig wird es durch ein $\backslash$r gekennzeichnet.}
%}
%\newglossaryentry{Zeilenumbruch}{
%name=Zeilenumbruch,
%description={Dieser Befehl besagt, das der Curser in der nächsten Zeile beginnen soll. Häufig wird es durch ein $\backslash$n gekennzeichnet. Bei Windowssystem inkludiert ein Zeilenumbruch auch immer einen Zeilenrücksprung durch $\backslash$r. Dieses wird bei Unixsystemen getrennt}
%}
%\newglossaryentry{Encodieren}{
%name=Encodieren,
%description={Ein Datenpaket (wie z.B. einen Text) mit einem Code verschlüsseln},
%plural=encodieren
%}
%\newglossaryentry{PDF Processor}{
%name=PDF Processor,
%plural=PDF Prozessoren,
%description={Ein PDF Processor ist jede Art von Software oder Hardware, welche PDF Dateien schreiben, lesen aktualisieren (updaten) oder anderweitig verarbeitet. Grundlegend wird zwischen \gls{PDF Writer} (PDF Software oder Hardware zum Schreiben von PDF Dateien) und PDF Readern (PDF Software oder Hardware zum Anzeigen von PDF Dateien) unterschieden.  }
%}
%\newglossaryentry{escape}{
%name=escape,
%plural=escaped,
%description={Unter escape (Escapen) versteht man in der Informationstechnik den Vorgang, eine Sonderfunktion, z.B. eines Zeichens, zu ignorieren und das Zeichen als normales Textzeichen darzustellen, anstatt die Sonderfunktion auszuführen. Hierzu wird wiederum ein Zeichen mit Sonderfunktion benötigt, welche als Escapezeichen bezeichnet wird. }
%}
%\newglossaryentry{case-sensitive}{
%name=case-sensitve,
%description={Unter case-sensitve versteht man, das die Groß und Kleinschreibung berücksichtigt wird \glqq Hallo\grqq\ entspricht somit nicht! dem Wort \glqq hallo\grqq\. Es sind 2 unterschiedliche Wörter. }
%}
%
%\newglossaryentry{PDF Writer}{
%name=PDF Writer,
%description={Ein PDF Writer ist eine Software oder Hardware, mit welcher PDF Dateien erstellt oder erzeugt (z.B. ein PDF Printer) werden können }
%}
%\newglossaryentry{PDF Reader}{
%name=PDF Reader,
%description={Ein PDF Reader (oder auch PDF Viewer) Software oder Hardware zum Anzeigen und Verarbeiten (\gls{PDF Converter}) von PDF Dateien. Abhängig vom Funktionsumfang des PDF Readers können mit diesem auch Formulare ausgefüllt, Inhalte extrahiert, Multimediainhalte wieder gegen werden oder Eingebettete Inhalte und Funktionen geöffnet oder Ausgeführt werden, wie das öffnen einer Internetseite oder eines Dokumentes oder das Ausführen von Programmen}
%}
%\newglossaryentry{PDF Converter}{
%name=PDF Converter,
%description={Software oder Hardware, welche PDF Dateien in andere Dateiformate umwandelt. Oder andere Dateiformate in PDF Dateien umwandelt.}
%}
%\newglossaryentry{PDF Printer}{
%name=PDF Printer,
%description={Ein virtueller Drucker (oder vielmer Druckertreiber), welcher empfangene Daten in PDF Syntax umwandelt und das Ergebniss als PDF Dateien abspeichert. Es wird somit ein PDF Dokument erzeugt, anstatt Inhalte Physikalisch auszudrucken.}
%}
%
%\newglossaryentry{PDF Drucker}{
%name=PDF Drucker,
%description={Ein virtueller Drucker (oder vielmer Druckertreiber), welcher empfangene Daten in PDF Syntax umwandelt und das Ergebniss als PDF Dateien abspeichert. Es wird somit ein PDF Dokument erzeugt, anstatt Inhalte Physikalisch auszudrucken.}
%}

\newglossaryentry{Meta}{
name=Metadaten,
description={Metadaten oder Metainformationen sind Informationen über andere Daten}
}

\newglossaryentry{Metadatendatei}{
name=Metadatendatei,
description={Eine Datei, welche \glspl{Meta} (Informationen zu anderen Daten) über ndere Daten enthält.}
}
%\newglossaryentry{End Of Line}{
%name=End Of Line,
%description={Symbolisiert das \glqq Ende einer Zeile\grqq\.}
%}
%\newglossaryentry{Token}{
%name=Token,
%description={PDF Dateien bestehen aus einer Sequenz von 8 Bit Binärbytes. Sie werden zu Gruppen zusammen gefasst die Schlüsselwörter, Strings, Numbers, usw repräsentieren. Eine Gruppe von Byte wird in PDF als Tokens definiert \cite[S. 48]{pdf}}
%}
%%\newglossaryentry{case-sensitive}{
%%name=case-sensitive,
%%description={case-sensitive bedeutet, das zwischen Klein und Großbuchstaben unterschieden wird. "\glqq Ein  Beispiel\grqq\ ist also \textbf{nicht!} gleich \glqq ein beispiel\grqq\ .}
%%}
%
%\newglossaryentry{Font}{
%name=Font,
%plural=Fonts,
%description={Als Font wird der Informationstechnik jede Schriftart bezeichnet, die auf einem Computer oder Peripheriegerät vorhandene ist.}
%}
%\newglossaryentry{decodieren}{
%name=decodieren,
%plural=decodiert,
%description={Eine Datei oder Nachricht wieder Entschlüsseln und ins ursprüngliche Format zurück übersetzten.}
%}
%\newglossaryentry{Offset}{
%name=Offset,
%description={Eine bestimmte Speicheradresse oder eine Anzahl Bits, Bytes, ..., die i.d.R vom Beginn einer Datei berechnet berechnet wird. }
%}
%\newglossaryentry{Hybride PDF Datei}{
%name=Hybride PDF Datei,
%plural=hybriden PDF Dateien,
%description={PDFs, die zusätzlich zu Objektstreams und einem \gls{XRef} Stream (verfügbar ab PDF Version 1.5), auch normale Objekte und eine normale \gls{XRef} Tabelle enthalten (bis PDF Version 1.4 die einzige Form). Siehe hierzu das Kabitel \glqq Compatibility with Applications That Do Not Support PDF 1.5\grqq\ der Adobe Referenz unter \cite[S. 109 ff]{pdf2}}
%}
%
%\newglossaryentry{Update}{
%name=Update,
%plural=Updates,
%description={Ein Update ist eine Aktualliserung. Etwas wird auf eine aktuellen Stand oder eine aktuelle Version gebracht. }
%}
%
%\newglossaryentry{inkrementell}{
%name=inkrementell,
%plural=inkrementelle,
%description={schrittweise erfolgend, aufeinander aufbauend. \cite[Schlagwort: inkrementell]{duden} }
%}
%
%
%\newglossaryentry{EOL}{
%name=End of Line,
%description={Mit \ac{EOL} wird das Ende einer Zeile beschrieben. Der Marker beschreibt in der Informatik häufig die Kombination der Befehle für eine neue Zeile (meist \bb n) und einen Zeilenrücksprung (meist \bb r)}
%}
%\newglossaryentry{EOF}{
%name=End of File,
%description={Mit \ac{EOF} wird das Ende einer Datei beschrieben. Wie dieser Marker aussieht ist abhängig vom Programm, welches eine solche Markierung zum Lesen einer Datei benötigt.}
%}
%\newglossaryentry{Hash}{
%name=Hash,
%plural=Hases,
%description={Ein Hash ist eine bezeichnet eine Verkürzte Form von etwas. Durch eine Hashfunktion (der Vorgang um einen Hash zu erstellen) wird eine Große Eingabe (z.B. eine Datei oder ein Satz) in eine kleine Ausgabemenge umgewandelt(abgebildet). Es existieren verschiedene Verfahren, um einen Hash (oder Hashwert) zu erzeugen. Die bekanntesten sind \ac{MD5} und \ac{SHA} in Version 2 und 3. Ein Hashwert wird häufig als Prüfsumme verwendet und wird in diesem Fall auch als Fingerprint bezeichnet.}
%}

\newglossaryentry{Tool}{
name=Tool, description={Ein Tool (Informationstechnik, zu Deutsch
    \textit{Werkzeug})  beschreib beschreibt ein Stück Software das
    Hilfsaufgaben ausführt um eingesetzt bei anderen Aufgaben zu unterstützen }}



%\glossarystyle{altlistgroup}
%\setglossarystyle{listhypergroup}
\printglossaries
\addcontentsline{toc}{chapter}{Glossar}  %after print
\cleardoubleemptypage

\mainmatter
\chapter{Einleitung} \label{chp:einleitung}


\section{Über Yocto, OpenEmbedded, poky, meta-daten und Bitbake}

Das Yocto Project bezeichnet eine Community Gruppe welche sich zur Aufgabe
gemacht hat, dass erstellen von Linux Distributionen für eingebettet System
zu vereinfachen.
\\

Zusammen mit der OpenEmbedded Community pflegt das \gls{Yocto Project} eine
Software Build Umgebung mit dem Namen \gls{Bitbake}, bestehend aus
Pythonskripten, welches das Erstellen von Linux Distributionen koordiniert.
Des Weiteren pflegen beiden Communities Dateien mit \gls{Metadaten}
(\gls{Metadatendatei} die in Form von Regeln  beschreiben wie Software Pakte
innerhalb für unterschiedliche Distribution und Hardware durch Bitbake gebaut
werden müssen.  erforderlich sind.

\ac{PDF}
\\

Diese Regeln  werden \gls{Recipes} genannt. Regeln bzw. diese Metadaten und
somit auch das Build-System Bitbake arbeiten nach einen Schichten Modell.
Dabei beschreiben Meta-Daten auf unterster Schicht allgemeine grundlegende
Anleitungen zum Übersetzten der wichtigsten Funktionen eines Betriebssystems.
Höhere Schichten erweitern (detaillieren) oder überschreiben diese grundlegenden
Rezepte in immer höheren Schichten. So setzt auf Beschreibungen der Hardware
(BSP, Board Support packed Schichten) Schichten der Software- und
Anwendungsschicht auf. Ein solches Schichtenmodell ermöglicht es, einzelne
Schichten auszutauschen oder darunterliegende Einstellungen abzuändern.
Die Yocto community stellt Meta-Daten / Recipes bereit, um ein
minimalistisches Betriebssystem mit grundlegenden  Linux Tools unter der
Virtualisierungsumgebung QEMU für verschiedene Architekturen starten zu können.
Die OpenEmbedded Community stellt Meta-Daten Rezepte bereit, die auf dieses
minimalistische Betriebssystem aufsetzten und genutzt werden können um ein
Betriebssystem nach eigenen Wünschen gestalten ( zusammen setzten) und für
Hardware Plattformen konfigurieren zu können. Hierzu gehören sowohl Hardware als
auch open Source Software Beschreibungen. Beide Communities pflegen und
erweitern das Build Umgebung „Bitbake“, welches verschiedene Aufgaben in
geregelter
Reihenfolge im Multicore Betrieb durchführt:

\begin{itemize}
    \item Herunterladen (fetchen) und Sammeln von Source Dateien
        (z.B. Git Repositorien, ftp Servern oder lokalen Dateien.)
    \item Übersetzten, konfigurieren, patchen, installieren, verifizieren usw.
        von Paketen auf mehreren Prozessor Kernen
    \item Bereitstellen von Entwicklung und Verwaltungstools
\end{itemize}

Das Minimalisten Betriebssystem der Yocto Community wird „poky“ genannt.


\section{Docker} \label{sec:docker}
Bitbake benötigt, neben einem aktuellen Python 2 Interpreter, verschiedene
Tools. So u.a. Git zum Herunterladen von Source Detain. Alle diese
Abhängigkeiten wurden in einem \gls{Docker} Container zusammengefasst.  Bitbake kann
unter unschädlichen Betriebssystemen und Plattformen auf gleiche Weise genutzt
werden, ohne dass es und seine Abhängigkeiten neu konfiguriert werden müssen.
Im Gegensatz zu anderen Virtualisierungstechniken hat Docker trotz
Virtualisierungstechniken keinen großartigen Performance Verlust.
Von einem Gebrauch von klassischen Virtualisierungstechniken ist aus
Performance Gründen dringend abzuraten.






\chapter{Bitbake}%
\label{cha:bitbake}

\section{Bitbake Buildprozess}%
\label{sec:bitbake_buildprozess}



Die Buildumgebung \textit{Bitbake}, im Kern bestehend aus Phython- und
Shellscripen die mit unterschiedlichen Softwareentwicklungstools wie GIT, Make
oder Autotools interagieren, führt in einer geregelten Reihenfolge verschiedene
Aufgaben aus. Die Abbildung auf \cite[S. 20]{Gonzalez2018:Embedded_Linux_Development_Using_Yocto_Project_Cookbook_2nd}
zeigt abstract den Build\-process:

\begin{itemize}
    \item Rhein\-folge der gepassten config files gepasst werden.
    \item Parsen der config files.
    \item Buildschritte (Siehe \ref{sec:bitbake_build_tasks} Seite \pageref{sec:bitbake_build_tasks}
    \item Packetieren in unterschiedliche Pakettypen
    \item Generieren und bereitstellen von \acfpl{SDK}
    \item Aufräum- und Nacharbeiten
\end{itemize}

\section{Bitbake build tasks}%
\label{sec:bitbake_build_tasks}

Jedes \textit{recipe} erbt eine Reihe von standard Build-Tasks.  Hierzu
gehören u.a.
\begin{itemize}
    \item Herunterladen (fetchen) und Sammeln von Source Dateien
        (z.B. Git Repositorien, ftp Servern oder lokalen Dateien.)
    \item Übersetzten, konfigurieren, patchen, installieren, verifizieren usw.
        von Paketen auf mehreren Prozessor Kernen
    \item Bereitstellen von Entwicklung und Verwaltungstools
\end{itemize}

Eine Auflistung der Standard-Task kann, mitsamt kurzer Beschreibung,
\cite[S. 171-172]{Gonzalez2018:Embedded_Linux_Development_Using_Yocto_Project_Cookbook_2nd}
entnommen werden.

\subsection{Fatch Task}%
\label{sub:fatch_task}
Eine der Hauptaufgaben von Bitbake besteht dadrinne, die jeweiligen
Softwarepakte in Ihren benötigten Versionen und Revisionsständen zusammen aus
unterschidlichen Quellen zusammen zu sammeln und dem Buildsystem zur Verfügung
zu stellen. Solche Quellen können sein:

\begin{itemize}
    \item Lokaler Bitbake-Download ordner/cache. Er enthält bereits einmal
        heruntergeladene Softwarepakete in jeweiligen Revisionständen.
    \item Lokaler Pfad zu Quelldateien; z.b. in einem eclipse oder QT Work\space
    \item Lokale Repository (z.B. lokales GIT Repository)
    \item Netz\-werkpfad zu einem Client oder Server im lokalen Netzwerk. Z.B.
        über Freigaben oder einem lokalen FTP Server
    \item Online Repository oder TFTP Server.
    \item Alternatives lokales oder Remote (online) Repository.
\end{itemize}

Die Reihe\-folge in welcher nach Source-Dateien gesucht werden soll ist zum
einen definiert durch:

\begin{itemize}
    \item das Recipe selbst welches das Software\-pakete innerhalb Bitbake bauen
        soll,
    \item durch Konfigurationsdateien wie \textit{./conf/local.conf),
    \tem sowie durch eine allgemein fest vorgegeben Reihe\-folge innerhalb
    bitbakes. (Siehe \cite[S.53]{Gonzalez2018:Embedded_Linux_Development_Using_Yocto_Project_Cookbook_2nd}
\end{itemize}



\section{Configurations-Dateien *.conf}%
\label{sec:configurations_dateien_conf}
Bitbake, sowie die Buildprozesse der einzelnen Recipes
werden gesteuert durch unterscheidliche \textbf{recipe-lokale} und
\textbf{globale} configurationsdateien.




%\chapter{Setup your host}%
\label{cha:setup_host}


\section{Erforderliche Pakete}%
\label{sec:erforderliche_pakete}

Das yocto Docker file zeigt eine Liste aller nötigen Ubuntu Pakte die zum
Arbeiten mit der Yocto / OpenEmbedded Build Umgebung auf einem  Host benötigt
werden, sollte nicht mit dem Docker Image gearbeitet werden wollen. \\


\textbf{Zusätzlich} sind die nachfolgenden Pakete werden zum Arbeiten auf dem lokalen
Host benötigt. Nähere Informationen zu den Paketen, Quellen, Konfigurationen
usw. sind auf verschiedenen Internetseiten zu finden.

\begin{itemize}
    \item git
    \item docker
    \item TFTP Server
    \item NFS Server
    \item microcom
    \item eclipse
    \item qt5
    \item qt5Creator
    \item openssh-server
\end{itemize}

\section{Host Konfiguration}%
\label{sec:host_konfiguration}

Nachfolgende Pakete benötigen weitere Konfigurationen
\begin{description}
    \item[Docker: ]
        \begin{itemize}
            \item[ ]
            \item Docker Service starten
            \item Docker-yocto Image bauen:
                \glqq docker build -t yocto . \grqq
            \item Hilfe liefert docker --help oder die Internetseite
            \item Image starten durch ausführen von \glqq ./run.sh bash \grqq
        \end{itemize}

    \item[TFTP Server:]
        \begin{itemize}
            \item[ ]
            \item TFTP Austausch Ordner anlegen und Zugriffsrechte definieren
            \item stat-alone deamon (/etc/default/tftpd-hpa) oder xinitd Service
                (/etc/xintd.d/tftp) konfigurieren
            \item Server neu starten
            \item \textbf{BEISPIEL} im \glqq BSP Manual\grqq unter  phytec.de;
                    Stichwort \glqq Booting the Kernel from Network\grqq
                (Booting\_the\_Kernel\_from\_Network) \cite{Pytec:BSP_Manual}
                oder unter \cite[S.
                44]{Gonzalez2018:Embedded_Linux_Development_Using_Yocto_Project_Cookbook_2nd}
        \end{itemize}

    \item[NFS Server:]
        \begin{itemize}
            \item[ ]
            \item NFS Server konfigurieren (/etc/exports)
            \item NFS Server neu starten
            \item \textbf{BEISPIEL} im \glqq BSP Manual\grqq unter  phytec.de;
                    Stichwort \glqq Booting the Kernel from Network\grqq
                (Booting\_the\_Kernel\_from\_Network) \cite{Pytec:BSP_Manual}
                oder unter \cite[S.
                45]{Gonzalez2018:Embedded_Linux_Development_Using_Yocto_Project_Cookbook_2nd}
        \end{itemize}
    \item[Microcom]
        \begin{itemize}
            \item[ ]
            \item Der Parameter --port Definiert die Serielle Schnittstelle.
            \item Weiteres ist unter Manual Seite zu finden.
        \end{itemize}
    \item[Eclipe]
        \begin{itemize}
            \item Weiters im Kapitel \ref{cha:setup_eclipse}; Seite
                \pageref{cha:setup_eclipse}
        \end{itemize}
    \item[QT5Creator]
        \begin{itemize}
            \item[ ]
            \item Weiters im Kapitel \ref{cha:setup_qtcreator}; Seite
                \pageref{cha:setup_qtcreator}
        \end{itemize}
    \item[Yocto Areitsverzeichnis]
        \begin{itemize}
            \item[ ]
            \item Erstellen eines globalen Arbeitsverzeichnisses.  Z.B.
                /opt/yocto \item Setzten der Rechte rwx Rechte für alle user
                (\glqq others\grqq).
        \end{itemize}
    \item[Python2 als standart interpreter] Yocto/Openembedded Tools bauen auf
        Python2 auf. Daher ist es nötig einen  symlink  oder alias
        auf Python2 zu setzten. Z.b. \textit{alias python=python2}
    \item[Proxy und Routen]
    Je nach Netzwerkinfrastruktur müssen Proxy und Netzwerkrouten auf
    dem lokalen Host gesetzt werden. Beispielsweise für die Tools:
        \begin{itemize}
            \item Git
            \item wget
            \item apt-get
            \item https\_proxy und http\_proxy
        \end{itemize}
        \textbf{Informationen} hierzu liefert die das Yocto Manual oder das
            Yocto Wiki unter dem Stichwort \glqq \textbf{Working Working Behind
            a Network Proxy} \grqq (Working\_Behind\_a\_Network\_Proxy)

\end{description}


%

########## Build setup #########################################################


================================================================================
1.  Download minimal basic project called "reference project" or "poky"

--------------------------------------------------------------------------------
1.1 For phytec download a script and run it in a new empty project folder:

    wget ftp://ftp.phytec.de/pub/Software/Linux/Yocto/Tools/phyLinux
    chmod +x phyLinux
    ./phyLinux init

    For more information see the phytec Manual
    https://www.phytec.de/documents/?title=l-813e-7-yocto-reference-manual#phyLinux


--------------------------------------------------------------------------------
1.2 For general use download the reference project poky directly

    git clone git://git.yoctoproject.org/poky


================================================================================
2.  Setup the build enviroment:

--------------------------------------------------------------------------------
2.1 Configure the following dictionary locations:

................................................................................
2.1.1 What?
    - deploy directory = each project should have its own deploy directory
                        where to put the compilation results.
    - sstate directory = The (temporary) building stats holding intermediate
                        files. This can be shared between multi project
    - download directory = This folder holds all files which were fetched
                            (downloaded) from the internet. This folder should
                            be globale to all project so that the files must
                            only downloaded once for all.


................................................................................
2.1.2 Where?
    But the following into your build/conf/local.conf file:
    - SSTATE_DIR = "<path/to/your/folder>"
    - DL_DIR = "<path/to/your/folder>"
    - DEPLOY_DIR = "<path/to/your/folder>"


--------------------------------------------------------------------------------
2.2 Source the configuration script into your current shell. It configures the
    build environment:

    soure poky/oe-init-build-env





================================================================================
3.0 Configure the build.

--------------------------------------------------------------------------------
3.1 Configuration files

................................................................................
3.2 Globle build files. Configuration for all meta-data and recipes

 (1) build/bblayers.conf    = Meta data layers to include into the build.
 (6) build/local.conf       = Global project specific configuration file.
 (4) build/site.conf        = Same like local.conf. But should encapsulate
                                the destination (site) access information like
                                proxy settings.
 (5) build/auto.conf        = Meta data layers to include into the build.

 p
................................................................................
3.2 meta-data recipies/layer dependet configuration/build files. 

 (2) source/meta-<yourMetaDataLayer>/conf/layer.conf     =
            Same like local.conf but only for this meta-layer.
 (7) source/meta-<yourMetaDataLayer>/conf/<machine>.conf =
            Configuration about the target (mcu, board,...).
 (8) source/meta-<yourMetaDataLayer>/conf/<distro>.conf  =
            Configuration to the target distribution to compile.

................................................................................
3.3 Bitbake configuration file

 (3) source/poky/meta/conf/bitbake.conf

................................................................................
3.4 Order in which the files gets include/parsed.

    The order is defined by the numbers above in 3.1 - 3.2. Depending on the
    assignment to the variables, the first (normal) or least one wins. For
    further information take a look into the bitbake syntax how to assign
    values to variables.








================================================================================
x.0 Default and Template configurations

--------------------------------------------------------------------------------
x. 1 When calling (sourcing) the oe-init-build-env script you can define some
    default/template configuration files for 'local.conf' and 'bblayer.conf'.
    To do so, set the folder where this local.conf.sample or bblayer.conf.sample
    file live to the enviroment variable "TEMPLATECONF" before sourcing the
    script oe-init-build-env script.

    export TEMPLATECONF=<mypath/folder>




http://downloads.yoctoproject.org/releases/eclipse-plugin/2.6.1/oxygen/

%
\chapter{QT5 einrichten}%
\label{cha:qt5_einrichten}

\section{Qt Creator einrichten}%
\label{sec:qt_creator_einrichten}


Wie sich QT zur Cross-Entwicklung und Remote debugging einrichten und in
Kombination mit einem bitbake-SDK nutzen lässt ist beschrieben unter

\begin{itemize}
    \item \cite[Seite
        269-276]{Gonzalez2018:Embedded_Linux_Development_Using_Yocto_Project_Cookbook_2nd}
    \item \cite [Working with Qt Creator]{PhyTec:Development_Guid}
\end{itemize}

\section{Arbeiten mit Qt Creator}%
\label{sec:arbeiten_mit_qt_creator}

Der Workflow, wie sie QT-Anwendungen mittels Qt Creator entwickeln und in
Bitbake einbinden lassen, ist beschrieben unter:
    \cite[Seite
        277-285]{Gonzalez2018:Embedded_Linux_Development_Using_Yocto_Project_Cookbook_2nd}



%


to add the modules into your kernel image root fs, you need to define one of the
following variables in your machine/<yourMachine.conf> configuration file

MACHINE_ESSENTIAL_EXTRA_RDEPENDS

MACHINE_ESSENTIAL_EXTRA_RRECOMMENDS

MACHINE_EXTRA_RDEPENDS

ACHINE_EXTRA_RRECOMMENDS 


If you like to load the module during startup, you also need to define the
following variable in your machine/<yourMachine.conf> file.

KERNEL_MODULE_AUTOLOAD 

For further information please grep the yocto mega manual in "Variables Glossary"


%\include{content/software_module}


################################################################################
########## Nice to know ########################################################
- An 'image' wich will be build is also described by recipes. So all commands
  you find next can be used for an image to build as well as for a recipes




################################################################################
########## Commands ############################################################

# create a Yocto (default) layer interactivly and add it to /build/bblayer.conf
# global active/used project layer configuration. You don't have do write 
# "meta" in front of"
yokto-layer create <myLayer>
bitbake-layers create-layer <mylayer>
bitbake-layers add-layer <layername>




#show all project layer
bitbake-layers show-layers



# show all recipes and the location layer it belongs to. You can grep for a kind
# of recipes type. 
bitbake-layer show-recipes [<recipes>]
bitbake-layer show-recipes [<recipes>] | grep image


#show all tasks for a recipes
bitbake -c listtasks <recipes>


#run Command <cmd>. Commands/tasks depend on the recipes.
bitbake -c listtasks <recipes>
bitbake -c <cmd> <recipes>


#print a bitbake/yokto environment variable (used and set inside the project)
bitbake -e <recipes> | grep ^<ENVVARIABLE>



#compile the recipes and jump into its temporary "working directory". Either
# by a normal development (bash) shell or a python shell.
bitbake -c devshell <recipes>
bitbake -c devpyshell <recipes>


#get the linux kernel version and provider used in a project
bitbake -e virtual/kernel | grep ^SRC_URI=
bitbake virtual/kernel -e | grep  "PREFERRED_PROVIDER_virtual/kernel"








%\chapter{Yocto devtool}%
\label{cha:yocto_devtool}

Des Yocto \textit{devtool} unterstützt dabei, neue recipes zu erstellen,
vorhandene anzupassen oder etwa Sourcedateien zu patchen.

\section{devtool Kommandos}%
\label{sec:devtool_kommandos}

Eine Übersicht über die Kommandos des Yocto \textit{devtool} liefert
\cite[Seite
245]{Gonzalez2018:Embedded_Linux_Development_Using_Yocto_Project_Cookbook_2nd}

\section{Arbeiten mit \textit{devtool}}%
\label{sec:arbeiten_mit_devtool}

Verschiedene \glspl{Workflow} zum arbeiten mit \textit{devtool} sind zu finden
unter \cite[Seite 239-249]{Gonzalez2018:Embedded_Linux_Development_Using_Yocto_Project_Cookbook_2nd}


%
\chapter{Bekannte Fehler - Pittfalls}%
\label{cha:bekannte_fehler}


\section{Never do}%
\label{sec:Never_do}

Nachfolgende actionen füren in jedem Fall zu einem Fehler

\begin{description}
    \item[Unterstrich \glqq \_ \grqq im Namen]  Vergebe niemals einen Namen mit
        einem Unterstrich ( \_ ); z.B. für ein Recipe oder einen Ordner
    \item[kein CamelCase] Namen für Recipies oder beispielsweise Ordner müssen
        klein geschreiben werden. CamelCase oder Großschreibung ist nicht
        zulässig.
    \item[Umbenennung von Dateien oder Pfaden] Benenne niemals einen Ordner,
        einen Pfad oder ähnliches um. Anderfalls ist ein komletter Neuanfang mit
        leerer \textit{Shell} und leerem \textit{temp}
        und \textit{sstate\_cache} Ordner nötig.
    \item Updaten, austauschen, \ldots von meta-layern, paketen, Konfigurationen
        \ldots. Neue Abhängigkeiten und veränderungen führen oft zu Fehlern.
\end{description}


\section{Must do}%
\label{sec:must_do}

\begin{itemize}
    \item Es \textbf{muss} der \textit{sstate chache} manuell gelöscht werden
        wenn:
        \begin{itemize}
            \item Kernel Konfigurationen verändert wurden. Z.B. durch
                \textit{menuconfig} oder \textit{bitbake -c configure virtual/kernel}
            \item Module zum Kernel-Autoload hinzu gefügt wurden. Siehe hierzu
                \fullref{sec:software_module}
            \item Eine \textbf{Maschinenkonfiguration}
                \textit{machine/mymachine.conf} verändert wurde.
            \item Eine \textbf{Distributionsconfiguration}
                \textit{distro/mydistro.conf} verändert wurde.
        \end{itemize}
    \item Sollte alles nichts helfen, muss auch der \textit{tmp} Ordner gelöscht
        werden oder ein \textit{bitbake --force <command\_and\_recipe}
        aufgerufen werden. Siehe herzu auch \fullref{sub:loesungsschritte}
\end{itemize}

\section{Generelle Empfehlungen}%
\label{sec:generelle_empfehlungen}

Beim Arbeiten mit Bitbake ist zu empfehlen:

\begin{itemize}
    \item Kleine Änderungen durchführen und testen.
    \item Eigene Meta-dateien und Änderungen in bestehenden Konfigurationen und
        Metadateien in einem Reposetorie Versionieren. In einem Fehlerfall ist
        so ein Rollback möglich.
    \item Sichere regelmässig local die Ordner
       \begin{itemize}
           \item \textit{sstate-cache}
           \item \textit{tmp}
           \item \textit{deploy} bzw. \textit{tmp/deploy}
       \end{itemize}
    \item Halte immer ein Lauffähiges \textit{Image}, \textit{rootfs},
    und einen funktionierenden
    \textit{device tree blob} und ggf. ein funktionierendes \textit{SDK}
    bereit; u.a. um  Fehler der Hardware ausschließen zu können.
    \item Vorgehensweise beim erstellen einer neuen Konfiguration:
        \begin{itemize}
            \item Eine Bestehende Image / Distro / Machine Configuration
                nutzen und nach eigenen Anforderungen abändern. Änderungen
                jeweils in kleinen schritten durchführen.
            \item Auf einer minimalen poky Konfiguration neu aufbauen. Ggf. an
                anderen Beispielen orientieren.
            \item Eine Bestehende größere Konfigurationsstruktur zu erweitern
                ist \textit{nicht empfehlenswert} da Fehleranfällig und schwer
                zu verstehen.
            \item Must-dos \fullref{sec:must_do} beachten.
            \item Never-do \fullref{sec:never_do} beachten.
        \end{itemize}
    \item Variablen sollten wie folgt erweitert werden:
        \begin{description}
            \item[In Konfigurationsdateien] durch die Operatoren \textit{\_append}
                oder \textit{\_prepend}.
            \item[In Recipes] durch Operatoren \textit{+=}
                oder \textit{.=}.
            \item[Weitere Informationen und Hintergründe unter] \cite[Seite
                160]{Gonzalez2018:Embedded_Linux_Development_Using_Yocto_Project_Cookbook_2nd}
        \end{description}
\end{itemize}




\chapter{Lösung für bekannte Fehler}%
\label{cha:losung_fur_bekannte_fehler}


\section{\textbf{ERROR:} Fehler beim Bauen eines recipes>}
\label{sec:fehler_beim_bauen_eines_images}


\subsection{lösung}%
\label{sub:losung_fehler_beim_bauen_eines_images}


\begin{enumerate}
    \item Run \textit{bitbake -c cleanall <recipes>}
    \item If this not work, clean (deleate) all:
    \begin{itemize}
        \item \textit{sstate} folder
        \item \textit{deploy} folder
        \item \textit{tmp} folder
        \item everythink else in the project folder, exapting / but not the conf
            folder
    \end{itemize}
\end{enumerate}

\chapter{Source code development und debugging}%
\label{cha:source_code_development}


\section{Cross development}%
\label{sec:cross_development}
Es existieren verschiedene Wege um Anwendungen für Ziel\-plattformen zu
entwickeln. Yocto setzt dabei auf die beiden nachfolgenden; zum einen die
Bereitstellung einer gekapselten Entwicklungsumgebung in Form eines \glspl{SDK},
zum anderen die Entwicklung unter Zuhilfenahme der Build\-umgebung Bitbake
mitsamt seiner Tools.  \\

Nachfolgend beide Wege im Überblick:

\subsection{Entwicklung mit dem Yocto SDK}%
\label{sub:entwicklung_mit_dem_yocto_sdk}
Dieser Weg dient der normalen Entwicklung von Software\-komponenten. Das Yocto
SDK liefert eine gekapselte Shell-Umgebung mit vordefinierten
Umgebung\-variablen sowie allen nötigen Header Dateien, Libraries usw. welche
auf der Ziel\-plattform vorhanden sein werden. Hierzu bildes es die
Ordner\-struktur (das s.g. \gls{rootfs}) der Ziel\-plattform in einem Ordner
ab. \\


\begin{description}
    \item[1. Shell Umgebungsvariablen laden] Um die SDK Umgebung zu nutzen ist
        es nötig, eine Konfigurationsdatei in die Shell zu laden.

\begin{lstlisting}[frame=single,language=bash,caption={Einrichten der SDK
        Umgebung}]
[user@host]/yoctopath $: source ./enviroment-setup-*.sh
\end{lstlisting}


    \item[2. Eclipse starten] Abhängig vom Anwendung\-fall, anschließend in der
        selben Shell Eclipse oder QTCreator starten.

\begin{lstlisting}[frame=single,language=bash,caption={Start von Eclipse in der
        zuvor konfiguierten Umgebung}]
[user@host]/yoctopath $: eclipse &
\end{lstlisting}

\end{description}



\subsection{Entwicklung mit Yocto Tools}%
\label{sub:entwicklung_mit_yocto_tools}
Diese Software Entwicklungs\-art dient mehr dazu Änderungen an existierendem
Sourcecode durchzuführen. Hierbei stehen zwei tools zur Verfügung:

\begin{description}
    \item[devtool <cmd> ] Zum einen steht das Yocto-Tool \textbf{devtool} mit
        verschiedene Kommandos bereit. Eine Übersicht über Kommandos und
        Anwendungs\-beispiele liefert die Yocto Web\-seite oder \textit{devtool
            --help}.
    \item[bibake -c devshell <recipie>] Erzeugt bzw. öffnet eine
        vorkonfigurierte Shell ensprechend den Metadaten eines \textit{recipie}
        zu einer Source Datei.
\end{description}



\section{Remote Debugging}%
\label{sec:remote_debugging}





\chapter{Ausblick}%
\label{cha:ausblick}




\section{Nächste Schritte}%
\label{sec:naeste_schritte}

\subsection{Entwicklungswerkzeuge als Docker Container}%
\label{sub:developmenthost}

Es wäre sinnvoll die Entwicklungstools in einem oder mehrere Docker
container vorkonfiguiert zur verfügug zu stellen. Hierzu zählen unter anderem:

\begin{description}
    \item[Eclipse] Vorkonfiguriertes Eclipse inclusieve Plugins, Cross-Compile
        und Remote debugging Einstellungen.
    \item[QT5] Vorkonfiguiertes QT mit Crosscompile und Remot Debugging
    \item TFTboot und NFSROOT Server in einem Container vorkonfiguriert bereitstellen
\end{description}


\subsection{Erweiterung der Scripte}%
\label{sub:erweiterung_der_scripte}

Das \textit{run.sh} script sollte so erweitert werden, das es \glq post\grq
oder \glq pre\grq Aufgaben vor oder nach dem aufrufen der \textit{dockerjobs.sh}
durchführt oder andere postbuild oder prebuild scripte aufruft. Denkbar wären:

\begin{itemize}
    \item Kopieren des zImages und \ac{DTB} in das \textit{TFT-boot} Verzeichnis
    \item Kopieren und extrahieren des rootfs in das nfs-rootfs Verzeichnis
\end{itemize}


Das \textit{run.sh} script erzeugt das \textit{dockerjobs.sh} Script sollte dem
run.sh script parameter übergeben werden. Anschließend startet das run.sh script
den Docker container in definierter version (gesetzt über Parameter oder direkt
innerhalb des run.sh scripts). Das dockerjobs.sh wird innerhalb docker durch das
image aufgerufen und enthält alle aufgaben welche durch den Container in
batchmode erfüllt werden sollen. Das dockerjobs.sh script lässt sich manuell
erweitern / erstellen. Es wird nur überschrieben, wenn dem run.sh
Ausführungs\-befehle übergeben werden.



\section{Security}%
Gerade zu Beginn der Entwicklung bietet sich an, zunächst auf viele
Sicherheit\-funktionen zu verzichten, da der gesamte Entwicklungs\-prozess
bereits komplex ist und ausreichend potentielle Fehlerquellen besitzt.
\\

Dennoch ist mindestens zum Ende eines Projektes, vor Veröffentlichung, das
Sicherheits\-konzept überarbeitet werden.
So müssen beispielsweise nachfolgende Themen bewertet und bearbeitet werden
\label{sec:security}

\begin{itemize}
    \item Linux Kernel härten
    \item SELinux
    \item Gesamt System härten
    \item SMACK
    \item Benutzer, Passwörter, Zugrifffsrechte, ACLs
    \item Netzwerkschnittstellen und Kommunikation absichern. Beispielweise
        durch verschlüsselte Datenübertragung
    \item Debugging, Flashing, Tracing Schnittstellen entfernen oder
        einschränken
\end{itemize}


\section{Lizenzen}%
\label{sec:lizenzen}

Eine wichtiges Thema ist die Lizensierung neuer Softwarekomponenten, welche
zusammen mit Open-Source paketen (i.d.R. mindestens dem Linux Kernel) genutzt
werden und mit diesen Kompatible sein muss. So muss ich über folgende
Kombinationen gedanken gemacht werden.

Lizenverwaltung und Kompatibilität von:
\begin{itemize}
    \item Neuen Sofwarekomponenten
    \item Bestehenden  Sofwarekomponenten
    \item Verwendeten / Eingebundenen Softwarepaketen, z.B. über genutzte
        meta-datan Layer aus Community Quellen.
\end{itemize}













%Bereites in Grundlagen includet.
%\input{./content/grundlagen/someFile}


% Schalgwortverzeichnis (Index)
\printindex



% Literaturverzeichnis
%http://be-jo.net/2013/08/latex-welchen-bibliographystyle-wahlen/
\singlespacing

\bibliography{bibtex}
\bibliographystyle{alphadin}  %Sortierung: Kürzel des Autors, Referenzierung: Kürzel des Autors und Jahreszahl (zweistellig), Deutsch, Darstellung im Literaturverzeichnis:  Autoren Kapitälchen, Titel kursiv
%
%addbibresource{bibtex} %% Einbinden der bib-Datei
%

%\bibliographystyle{abbrvdin*}  %Sortierung: Name des Autors, Referenzierung: Nummer, Deutsch, Darstellung im Literaturverzeichnis: Autor in Kapitälchen, Initialen des Vornamens, Titel in kursiv
%\bibliographystyle{abbrvdin*}  %Sortierung: Name des Autors, Referenzierung: Nummer, English, Darstellung im Literaturverzeichnis: Vorname als Initialen, Titel kursiv.
%\printbibliography[type=article,title={Articles}]
%\printbibliography[type=book,title={Books}]
%\printbibliography[type=Webpage,title={Online Webpages}]
%
%\printbibliography[keyword={physics},title={Physics-related only}]
%\printbibliography[keyword={latex},title={\LaTeX-related only}]
%


% Eidesstattliche Erklärung
% \include{content/affirmation}

\appendix
% Hier können Anhaenge angefuegt werden

\end{document}
