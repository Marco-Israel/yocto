%
%********************************************************************************
%*** HOW to add existing software to the build -  Aka how to add a layer
%********************************************************************************
%
%
%--------------------------------------------------------------------------------
%Websites and Locations:
%
%
%(1) Official yokto git repo:
%    https://git.yoctoproject.org/
%
%
%(2) Description of the yokto meta data layer and its content
%        http://layers.openembedded.org/layerindex/branch/master/layers/
%
%(3)
%
%
%--------------------------------------------------------------------------------
%How to get your distro version (branch)
%
%    bitbake -e <recipes/image> | grep ^LAYERSERIES_COMPAT
%
%
%
%--------------------------------------------------------------------------------
%Find the meta layer which descries (is holding) your packet.
%
%
%- Go to link (1),
%
%- select your distribution branch,
%
%- configure a search filter if possible (like "software" layer only). So only
%    meta-data layer wich descries software packages will be displayed.
%
%- select the "respire" search option
%
%- After preforming / configure the search options, search for the software
%    packet you like to include into your build.
%
%- You will get the name of the meta-data layer you need to download and add to
%    your build (see below)
%
%
%--------------------------------------------------------------------------------
%How to add the metadata-layer to your build system which integrate (burns)
%the software packed into your distribution.
%
%- search for the metadata-layer folder in your project folder to know if you
%    need to download (git clone) it before. Otherwise skip the next step
%
%- If the meta-data layer is not in the build already, download (git clone)
%    the metadata-layer into your meta-data folder which already holds the
%    minimal standard meta-data layer.
%
%        git clone <url>
%
%    Afterwards check out the branch wich is compatible to your distribution
%    branch version. "git branch -r" will show you the available branches of the
%    online repo.
%
%        git branch -r
%        git checkout <your-branch>
%
%
%- add the meta-data layer to your build/bblayer.conf configuration file.
%
%        BBLAYER += "<path/to/your/meta-data-layer>"
%
%    You can also use a bitbake command to automatically add the layer to
%    build/bblayer.conf:
%
%        bitbake-layers add-layer <path/to/your/meta-data-layer>
%
%    You can check if the metadata layer is in the build now/already or if
%    there are errors inside the .conf file or inside the layer. Therefor call:
%
%        bitbake-layers show-layers
%
%    NOTE: It is also an error if the layer has dependencies to other layers wich
%    are not in the build already. So a call to "bitbake-layers show-layers" can
%    also show you such dependence errors.
%
%- add the software packed name you like to build to your build/local.conf file.
%    Attention: It is important to have leading white space " " inside the quotes
%    before the packed name.
%
%        IMAGE_INSTALL_append = "< ><your-sw-packed>"
%        IMAGE_INSTALL_append = "< ><your-sw-packed1> <your-sw-packed2> <...>"
%
%    NOTE: _append is an "operator" like += wich preforms on the variable in
%    front of it. You can also use:
%
%        IMAGE_INSTALL += "<your-sw-packed>"
%        IMAGE_INSTALL .= "< ><your-sw-packed>"
%
%    Different to the classic operators _append or _pretend are executed after
%    all recipes are parsed by bitbake. The classic operators will be preformed
%    during the parsing. As a result _append will be preformed after classic
%    += or .= operators.
%
%    ATTENTION: Using the operators +=, =+, .= or =. can result in errors because
%    it is not save in which order (moment) the variable get expend. It is
%    recommended by yokto to use the operator _append to make sure the Variable
%    will be appended at least after parsing the precipice.
%
%    You also can install a software only for a specific image. Doing so:
%
%        IMAGE_INSTALL_append_<your-image-recipes> = "< ><your software packed>
%        IMAGE_INSTALL_<your-image-recipes> += "<your software packed>
%
%    Alternatively you can modify or patch the IMAGE_INSTALL variable
%    in <your-image-recipes> by adding the software-packed the same way like
%    above.
%
%    It is also possible to create and/or add a packedgroup to your image which
%    predefined a group of software packed wich should be installed together.
%    More information about packed groups you find in the yokto-mega-manual.
%    Simple grep for "packedgroup" through the document.
%
%
%--------------------------------------------------------------------------------
%Finally
%    (Re-)build the image.
%
%        bitbake <your-image-recipe>
%
%
%
%
%
%
%
%
%
%--------------------------------------------------------------------------------
%
