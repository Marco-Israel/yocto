\chapter{Applikationsentwicklung}
\label{chap:applikationsentwicklung}

\section{Einbinden von Applikationen in vorhandene Images}%
\label{sec:einbinden_von_applikationen_in_vorhandene_images}

Die nachfolgende Schritte sind nötig um ein existierendes Softwarepaket zu einem
Image hinzu zu fügen.

\section{Existierende Pakete und Metadaten nutzen}%
\label{sec:existirende_pakete_und_metadaten}

\begin{enumerate}
    \item Image Branches herausfinden:
        \textit{bitbake -e <recipes/image> | grep \^LAYERSERIES\_COMPAT}

    \item Seite \cite{OpenEmbedded:Indexer_Recipes} aufrufen

    \item \textbf{Branch} auswählen und ggf. Reiter \textbf{recipies} neu
        auswählen.

    \item \textbf{Suchen} nach dem gewünschten Software\-paket in der Suchleiste.
        Dies liefert als Ergebnis den jeweiligen Metadaten Layer, welcher das
        Recipes enthält.

    \item Verifizieren, ob der Metadaten-layer bereits existiert: Z.b. durch
        \begin{itemize}
            \item Manuelles Suchen in der lokalen (meta-layer) Verzeichnisstruktur
            \item \textit{find --name <new-meta-layer>}
            \item \textit{bitbake-layers show-layer | grep <<new-meta-layer>}
        \end{itemize}

    \item Layer Packet herunterladen in das lokale Yocto Arbeitsverzeichnis, in
        welchem sich bereits \textit{Poky} und andere OpenEmbedded
        Metadaten-Layer befinden.:
        \begin{itemize}
            \item \textit{Manuelles} Herunterladen und
                \textit{einfügen in ./conf/bblayersi.conf}
                BBLAYER += "<pfad/zum/meta-data-layer>"
            \item \textit{git clone <url-to-meta-layer>} und manuelles einfügen
                                        in \textit{bblayersi.conf}
            \item \textit{bitbake-layers layerindex-fetch <layer-name>}
        \end{itemize}

    \item Beim manuellen Herrunterladen, den neuen Layer zu
        \textit{./conf/bblayers.conf} hinzufügen. Dieser
        Schritt wird durch \textit{bitbake-layers layerindex-fetch
            <metadatalayer-name>} bereits vollzogen.
    \item Abhängigkeiten Metadaten-layer herunterladen und ebenfalls dem Projekt
        hinzufügen. Abhängigkeiten lassen sich ermitteln oder ausgeben durch:
        \begin{itemize}
            \item \textit{anklicken des Recipie
                    names} unter \cite{OpenEmbedded:Indexer_Recipes} einsehen
                    und von dort, wie oben beschriebe manuell oder tool gestützt
                    herunterladen
            \item direkt ausgeben durch:
                \textit{bitbake-layers layerindex-show-depends <layer-name>}
                und anschließend herunterladen und hinzufügen durch
                \textit{bitbake-layers layerindex-fetch <layer-name>}
        \end{itemize}
    \item Falls nicht im Vorfeld ausgewählt, den nötigen
        \textit{Branch auschecken} mit \textit{git checkout <branch>} in jedem
        neuen Metadaten-layer.
    \item Auf Fehler überprüfen (z.B. auf fehlende Abhängigkeiten):
        \textit{bitbake-layers show-layers}
\end{enumerate}

\subsection{Sotware dem Image hinzufügen}%
\label{sub:sotware_dem_image_hinzufugen_}
Um ein Softwarepaket einem Image hinzu zu fügen muss die Variable
\textit{IMAGE\_INSTALL} gesetzt oder erweitert werden. Hier gilt zu entschieden
ob das Hinzufügen \textit{global} für alle Image Recipes oder \textit{image-lokal}
für gezielt einzelne Images (sprich Distributionen) geschehen soll:
\begin{itemize}
    \item Setzten oder erweitern in \textit{./conf/loccal.conf}
    \item Abändern des <image-recipes>.bb files durch setzten / erweitern der
        Variablen
    \item Image erweitern durch erstellen eines <image-recipes>.bbappend files
    \item Image Variable gezielt für ein Image/Target erweitern (s.u.)
\end{itemize}

Um \textbf{eine Variable zu erweitern} bieten sich folgende mögliche Operatoren an :

\begin{description}
    \item[spätes Anhängen] IMAGE\_INSTALL\_append = "< ><your-sw-packed>"
    \item[spätes Anhängen] IMAGE\_INSTALL\_prepend += "<your-sw-packed>"
    \item[direktes Anhängen] IMAGE\_INSTALL += "<your-sw-packed>"
    \item[direktes Anhängen] IMAGE\_INSTALL .= "< ><your-sw-packed>"
    \item[mehrere Pakete direkt anhängen]
        IMAGE\_INSTALL\_append = "< ><your-sw-packed1> <your-sw-packed2> <...>"
    \item IMAGE\_INSTALL\_prepend += "<your-sw-packed1> <your-sw-packed2> <...>"

    \item IMAGE\_INSTALL\_append\_<your-image-recipes> = "< ><your software packed>
    \item IMAGE\_INSTALL\_<your-image-recipes> += "<your software packed>
\end{description}

\textbf{ACHTUNG:} \\
Es sind die Varianten mit Operator \textbf{ \textit{append} und
\textit{prepend} vorzuziehen}!\\

\textbf{append} und \textbf{prepend} werden verarbeitet \textbf{nachdem} alle
Recipes von Bitbake geparsed und erstmalig verarbeitet wurden. Die Operatoren
\textit{+=} und \textit{-=} werden direkt beim Parsen eines Recipes verarbeitet.
Aufgrund des unterschiedlichen (früheren) Bearbeitungszeitpunktes kann bei den
Operaturen \textit{+=} und \textit{.=} nicht garantiert werden, zu welchem
Zeitpunkt und in welcher Reihenfolge eine Variable erweitert wird.
\textbf{Dieses kann in einigen Fällen zu Fehlern führen.} Siehe hierzu auch
\cite{Gonzalez2018:Embedded_Linux_Development_Using_Yocto_Project_Cookbook_2nd}. \\

Generelle Unterscheidung:
\begin{description}
    \item[\_append \_prepend] für .conf Dateien. Diese Dateien werden vor
        recipes verarbeitet. Durch diese Operatoren wird dennoch erreicht, das
        die Konfiguration abschließend Anwendung findet
    \item[ operatoren += .= ..] Sind für recipes
\end{description}

\textbf{ACHTUNG ACHTUNG: beachte das < >}, welches ein i\textbf{führendes
Leerzeichen} symbolisiert und bei dieser Art der Variablenerweiterung
\textbf{zwingend} erforderlich ist.







%********************************************************************************
%*** HOW to add existing software to the build -  Aka how to add a layer
%********************************************************************************
%
%
%--------------------------------------------------------------------------------
%Websites and Locations:
%
%
%(1) Official yokto git repo:
%    https://git.yoctoproject.org/
%
%
%(2) Description of the yokto meta data layer and its content
%        http://layers.openembedded.org/layerindex/branch/master/layers/
%
%(3)
%
%
%--------------------------------------------------------------------------------
%How to get your distro version (branch)
%
%    bitbake -e <recipes/image> | grep ^LAYERSERIES_COMPAT
%
%
%
%--------------------------------------------------------------------------------
%Find the meta layer which descries (is holding) your packet.
%
%
%- Go to link (1),
%
%- select your distribution branch,
%
%- configure a search filter if possible (like "software" layer only). So only
%    meta-data layer wich descries software packages will be displayed.
%
%- select the "respire" search option
%
%- After preforming / configure the search options, search for the software
%    packet you like to include into your build.
%
%- You will get the name of the meta-data layer you need to download and add to
%    your build (see below)
%
%
%--------------------------------------------------------------------------------
%How to add the metadata-layer to your build system which integrate (burns)
%the software packed into your distribution.
%
%- search for the metadata-layer folder in your project folder to know if you
%    need to download (git clone) it before. Otherwise skip the next step
%
%- If the meta-data layer is not in the build already, download (git clone)
%    the metadata-layer into your meta-data folder which already holds the
%    minimal standard meta-data layer.
%
%        git clone <url>
%
%    Afterwards check out the branch wich is compatible to your distribution
%    branch version. "git branch -r" will show you the available branches of the
%    online repo.
%
%        git branch -r
%        git checkout <your-branch>
%
%
%- add the meta-data layer to your build/bblayer.conf configuration file.
%
%        BBLAYER += "<path/to/your/meta-data-layer>"
%
%    You can also use a bitbake command to automatically add the layer to
%    build/bblayer.conf:
%
%        bitbake-layers add-layer <path/to/your/meta-data-layer>
%
%    You can check if the metadata layer is in the build now/already or if
%    there are errors inside the .conf file or inside the layer. Therefor call:
%
%        bitbake-layers show-layers
%
%    NOTE: It is also an error if the layer has dependencies to other layers wich
%    are not in the build already. So a call to "bitbake-layers show-layers" can
%    also show you such dependence errors.
%



%- add the software packed name you like to build to your build/local.conf file.
%    Attention: It is important to have leading white space " " inside the quotes
%    before the packed name.
%
%        IMAGE_INSTALL_append = "< ><your-sw-packed>"
%        IMAGE_INSTALL_append = "< ><your-sw-packed1> <your-sw-packed2> <...>"
%
%    NOTE: _append is an "operator" like += wich preforms on the variable in
%    front of it. You can also use:
%
%        IMAGE_INSTALL += "<your-sw-packed>"
%        IMAGE_INSTALL .= "< ><your-sw-packed>"
%
%    Different to the classic operators _append or _pretend are executed after
%    all recipes are parsed by bitbake. The classic operators will be preformed
%    during the parsing. As a result _append will be preformed after classic
%    += or .= operators.
%
%    ATTENTION: Using the operators +=, =+, .= or =. can result in errors because
%    it is not save in which order (moment) the variable get expend. It is
%    recommended by yokto to use the operator _append to make sure the Variable
%    will be appended at least after parsing the precipice.
%
%    You also can install a software only for a specific image. Doing so:
%
%        IMAGE_INSTALL_append_<your-image-recipes> = "< ><your software packed>
%        IMAGE_INSTALL_<your-image-recipes> += "<your software packed>
%
%    Alternatively you can modify or patch the IMAGE_INSTALL variable
%    in <your-image-recipes> by adding the software-packed the same way like
%    above.
%
%    It is also possible to create and/or add a packedgroup to your image which
%    predefined a group of software packed wich should be installed together.
%    More information about packed groups you find in the yokto-mega-manual.
%    Simple grep for "packedgroup" through the document.
%
%
%--------------------------------------------------------------------------------
%Finally
%    (Re-)build the image.
%
%        bitbake <your-image-recipe>
%
%
%
%
%
%
%
%
%
%--------------------------------------------------------------------------------
%
