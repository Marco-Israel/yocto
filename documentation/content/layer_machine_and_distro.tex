

\chapter{Hardware und Linux distribution}%
\label{sec:hardware_und_linux_distribution}

Jeder Metadaten-layer aber auch jedes einzelne Recipe lässt sich gezielt
Hardware Beschreibungen bzw Hardware Gruppen oder einer (eigenen)
\textit{Linux-\gls{Distributionen}} zuordnen, womit jener Meta-Layer oder jenes
Revipe nur für solche Hardware oder Distribution Anwendung findet.

Die \textit{./conf/local.conf} definiert für einen Buildprozess die zu
verwendende Hardware Gruppe und Linux Distribution.

\section{Hardware (\textit{machine}) Beschreibung}%
\label{sec:hardware_machine_beschreibung}

Damit ein Linux System und sine Software Komponenten auf einer Ziel Hardware
lauffähig sind, muss diese Hardware sowohl für das Linux Betriebssystem, als
auch für Bitbake beschrieben werden. Dieses geschieht mittels
einer <hw-metalayer>/conf/machine/<machine>.conf Konfigurationsdatei.

\\

So muss beispielsweise einem Linux Kernel mittels eines \glsl{DTB} die Hardware
beschrieben bzw. bekannt gemacht werden. Bitbake muss unter anderem wissen
über welche s.g. \textit{\glspl{Feature} eine Hardware verfügt.
Ein Feature ist beispielsweise die Existenz eines Displays oder  einer nicht-
standard Pheripheral wie eine PCI-Port oder wifi.

\\

Eine lister möglicher \textit{Machine Features} ist zu finden unter:
\begin{itemize}
    \item \cite[Abschn. Machine_Features]{Yocto:Reference_Manual}
    \item \cite[Abschn. Machine_Features]{Yocto:Mega_Manual}
\end{itemize}

Zudem definiert die Maschinenkonfiguration
\begin{itemize}
    \item welche Linux distribution verwendet werden soll bzw. kann. Siehe
        hierzu \ref{sec:linux_distribution_definieren}, Seite
        \pageref{sec:linux_distribution_definieren}.
    \item Welcher Bootloader Anwendung finden soll. Siehe hierzu
        \ref{sec:bootloader_definierenr} Seite
        \pageref{sec:bootloader_definieren}.
    \item Welche Softwarepakete, Treiber, Kernelmodule, Konfigurationen, usw.
        für eine Zielplattform zwingend zum starten erfoderlich oder etwa
        wünschenswert sind.
\end{itemize}

Die Maschinenkonfiguration ist somit eine zusammenfassung aller nötigen und
optionalen Komponenten, damit eine Software mitsamt Betriebssystem auf einer
Zielhardware lauffähig und verwendbar wird.

\\

Sollte es Änderungen oder Erweiterungen an einer bestehenden Hardware-Platform
geben, so müssen unter Umständen auch diese bekannt gemacht werden. Hierzu kann
entweder eine bestehende Beschreibung verändert werden, oder eine neue
Maschinenbeschreibung erstellt werden. \cite[S.
91-95]{Gonzalez2018:Embedded_Linux_Development_Using_Yocto_Project_Cookbook_2nd}


\section{Linux Distribution definieren}%
\label{sec:linux_distribution_definieren}
Eines der \textbf{grundlegenden Ziele} von Bitbake, bzw. Ziel der Yocto und
OpenEmbedded Community ist die Möglichkeit eigene Linux Distributionen gezielt
für einen Anwendungsfall zu definieren. Dank seines  Schichtenmodells ist eine
Yocto-Distribution unabhängig von einer Hardware. Eine Distribution wird für
einzelne Hardwareplatformen immer neu übersetzt, beinhaltet jedoch die selben
Tools, Anwendungen, Konfigurationen oder sonstigen Features, insofern die
jeweilige Ziel-Hardware diese Features in Hardware unterstützt).



\section{Bootloader definieren}%
\label{sec:bootloader_definieren}
Ein bootloader ist ein Stück Software das die Schnittstelle zwischen Software
und Hardware darstellt. Er hat die Aufgabe ein Betriebssystem oder baremetal
Software zu starten oder zu aktualisieren. Er bestimmt, etwa wo der Stiegspunkt
einer Software oder eines Betriebssystems zu finden ist (Speicherkomponente,
Sektor, \ldots) und übernimmt die Aufgabe solche Software über eine definierte
Schhnittstelle zu aktualisieren.
\\
Einem Betriebssystem liefert es u.a. eine Beschreibung der Hardware in Form
eines \gls{DTB}.
\\
Yocto / OpenEmbedded setzt i.d.R. auf einen der nachfolgenden verfügbaren

\begin{itemize}
    \item U-Boot
    \item Barebox (eine Neuauflage des U-Boot Bootloaders)
\end{itemize}

Änlich wie die Hardware und Zielplattform (Siehe \ref{sec:hardware_machine_beschreibung}
Seite \pageref{sec:hardware_machine_beschreibung}) sowie eine
Betriebssystem\-distribution muss auch der Bootlooder mittels
Configurationsdateien und Recipes in der Bitbake Buildumgebung für eine
Zielplattform definiert werden.
\\
Nötige konfigurationen sind beispielhaft beschrieben unter \cite[93-95, 98-99,
100-108]{Gonzalez2018:Embedded_Linux_Development_Using_Yocto_Project_Cookbook_2nd}
und gehen i.d.R. mit der Maschinen definition einher.
