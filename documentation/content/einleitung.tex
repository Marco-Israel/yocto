\chapter{Einleitung} \label{chp:einleitung}


\section{Über Yocto, OpenEmbedded, poky, meta-daten und Bitbake}

Das Yocto Project bezeichnet eine Community Gruppe welche sich zur Aufgabe
gemacht hat, dass erstellen von Linux Distributionen für eingebettet System
zu vereinfachen.  \\

Zusammen mit der OpenEmbedded Community pflegt das \gls{Yocto Project} eine
Software Build Umgebung mit dem Namen \gls{Bitbake}, bestehend aus Python- und
Shell-Scripten, welches das Erstellen von Linux Distributionen koordiniert.
Parallel zu Bitbake  entstehen und wachsen innerhalb der Yocto Community
verschiedene tools wie z.B. Wrapper, welchen das Arbeiten mit Bitbake
vereinfachen sollen, indem diese Tools beispielsweise einzelne \glspl{Workflow}
abbilden.\\

Des Weiteren pflegen beiden Communities Dateien mit \gls{Metadaten}
(\gls{Metadatendatei} die in Form von Regeln  beschreiben wie Software Pakte
innerhalb für unterschiedliche Distribution und Hardware durch Bitbake gebaut
werden müssen.  erforderlich sind.  \\

Diese Regeln  werden \gls{Recipes} genannt. Regeln bzw. diese Metadaten und
somit auch das Build-System Bitbake arbeiten nach einen Schichten Modell.
Dabei beschreiben Meta-Daten auf unterster Schicht allgemeine grundlegende
Anleitungen zum Übersetzten der wichtigsten Funktionen eines Betriebssystems.
Höhere Schichten erweitern (detaillieren) oder überschreiben diese grundlegenden
Rezepte in immer höheren Schichten. So setzt auf Beschreibungen der Hardware
(BSP, Board Support packed Schichten) Schichten der Software- und
Anwendungsschicht auf. Ein solches Schichtenmodell ermöglicht es, einzelne
Schichten auszutauschen oder darunterliegende Einstellungen abzuändern.
Ein Vereinfachtes Modell ist abgebildet unter
\cite[S.26]{Gonzalez2018:Embedded_Linux_Development_Using_Yocto_Project_Cookbook_2nd}
\\
\subsection{Die Yocto Community}%
\label{sub:die_yocto_community}


Die Yocto community stellt Meta-Daten / Recipes bereit, um ein
minimalistisches Betriebssystem mit grundlegenden  Linux Tools unter der
Virtualisierungsumgebung QEMU für verschiedene Architekturen starten zu können.
\\

Das Minimalisten Betriebssystem der Yocto Community wird „poky“ genannt.
\\

Zudem Pflegt es zusammen mit der OpenEmbedded community das Buildsystem
\textit{bitbake}.

\subsection{Die OpenEmbedded Community}%
\label{sub:die_openembedded_community}


Die OpenEmbedded Community stellt Meta-Daten Rezepte bereit, die auf dieses
minimalistische Betriebssystem aufsetzten und genutzt werden können um ein
Betriebssystem nach eigenen Wünschen gestalten ( zusammen setzten) und für
Hardware Plattformen konfigurieren zu können. Hierzu gehören sowohl Hardware als
auch open Source Software Beschreibungen.Zusammen mit der Yocto Community
 pflegen und erweitern das Build Umgebung \textit{Bitbake}, welches
 verschiedene Aufgaben in geregelter Reihenfolge im Multicore Betrieb
 durchführt. Weiteres in \ref{bitbake_buildprozess}, Seite
 \pageref{bitbake_buildprozess}.




\section{Docker} \label{sec:docker}
Bitbake benötigt, neben einem aktuellen Python (2) Interpreter, verschiedene
Tools. So u.a. Git zum Herunterladen von Source Detain. Alle diese
Abhängigkeiten wurden in einem \gls{Docker} Container zusammengefasst. Bitbake
kann unter unschädlichen Betriebssystemen und Plattformen auf gleiche Weise
genutzt werden, ohne dass es und seine Abhängigkeiten neu konfiguriert werden
müssen.  Im Gegensatz zu anderen Virtualisierungstechniken hat Docker trotz
Virtualisierungstechniken keinen großartigen Performance Verlust.  Von einem
Gebrauch von klassischen Virtualisierungstechniken ist aus Performance Gründen
dringend abzuraten.



