\chapter{Einleitung} \label{chp:einleitung}


\section{Über Yocto, OpenEmbedded, Poky, Meta-daten und Bitbake}

\textit{Das \gls{Yocto} Project} bezeichnet eine Community Gruppe welche
sich zur Aufgabe gemacht hat, dass erstellen von Linux Distributionen für
eingebettet System zu vereinfachen. Dabei werden Distributionen gezielt für
einen Anwendungsfall zusammengestellt und für die jeweilige Zielhardware von
Grund auf direkt aus den Quelldateien übersetzt. Letzteres ergibt sich aus der
Vielfalt der unterschiedlichen Hardware und CPU bzw. Controller Architekturen.\\

Zusammen mit der \textit{\gls{OpenEmbedded} Community} pflegt das Yocto Project
eine Software Build Umgebung mit dem Namen \textit{\gls{Bitbake}}, bestehend aus
Python- und Shell-Scripten, welches das Erstellen von Linux Distributionen
koordiniert. Parallel zu Bitbake entstehen und wachsen innerhalb der Yocto
Community verschiedene \glspl{Tool} wie z.B. \gls{Wrapper}, welchen das Arbeiten
mit Bitbake vereinfachen sollen. \\

Des Weiteren pflegen beiden Communities \gls{Metadaten} (\gls{Metadatendatei}
die in Form von Regeln beschreiben, wie Software Pakte für unterschiedliche
Distribution und Hardware durch Bitbake gebaut werden müssen. Diese Regeln
werden in einer Python und GNU Make ähnlichen, eigenen, Bitbake Scriptsprache
geschrieben. Sie werden \gls{Recipes} (Rezepte) genannt.  \\

Regeln bzw. diese Metadaten und somit auch das Build-System Bitbake arbeiten
nach einen \textit{Schichten Modell}. Dabei beschreiben Meta-Daten auf unterster
Schicht allgemeine grundlegende Anleitungen zum Übersetzten der wichtigsten
Funktionen eines Betriebssystems, aufbauend auf Schichten welche zunächst die
Hardware oder den Bootloader beschreiben (\acf{BSP} Schichten). Höhere Schichten
erweitern (detaillieren) oder überschreiben grundlegende, tiefere Rezepte.  So
setzt auf Beschreibungen der Hardware Software- und Anwendungsschicht auf. Ein
solches Schichtenmodell ermöglicht es, einzelne Schichten auszutauschen oder
darunterliegende Einstellungen abzuändern. Ziel ist es, durch dieses
Schichtenmodell eine eigene Linux Distribution zu erschaffen.  \\

Ein Vereinfachtes Modell ist abgebildet unter
\cite[S.26]{Gonzalez2018:Embedded_Linux_Development_Using_Yocto_Project_Cookbook_2nd}



\subsection{Die Yocto Community}% \label{sub:die_yocto_community}

Die Yocto community stellt Meta-Daten (Recipes) bereit, um ein minimalistisches
Betriebssystem mit grundlegenden  Linux Tools unter der Virtualisierungsumgebung
\textit{\gls{QEMU}} für verschiedene Architekturen starten zu können.  \\

Das Minimalisten Betriebssystem der Yocto Community wird \textit{\gls{Poky}}
genannt.  \\

Zudem Pflegt es zusammen mit der OpenEmbedded community das Buildsystem
\textit{bitbake}.

\subsection{Die OpenEmbedded Community}%
\label{sub:die_openembedded_community}


Die OpenEmbedded Community stellt Meta-Daten Rezepte bereit, die auf dieses
minimalistische Betriebssystem aufsetzten und genutzt werden können um ein
Betriebssystem nach eigenen Wünschen gestalten ( zusammen setzten) und für
Hardware Plattformen konfigurieren zu können. Hierzu gehören sowohl Hardware als
auch open Source Software Beschreibungen.Zusammen mit der Yocto Community
 pflegen und erweitern das Build Umgebung \textit{Bitbake}, welches
 verschiedene Aufgaben in geregelter Reihenfolge im Multicore Betrieb
 durchführt. Weiteres in \fullref{sec:bitbake_buildprozess}.




\section{Docker} \label{sec:docker} Bitbake benötigt, neben einem Python
Interpreter, verschiedene Tools beispielsweise \textit{GIT} oder \textit{wget}.
Diese Tools, wie auch Python, sind in ihrer Version abhängig von der Bitbake
Version sowie den Metadaten. Es ist \textbf{dringend} erforderlich den
Versionsstand dieser Tools \glqq Einzufrieren \grqq. \\



An dieser Stelle kommt \textit{\gls{Docker}} ins Spiel. Alle Abhängigkeiten und
Versions\-stände in einem \gls{Docker} Container zusammengefasst ermöglicht das
Einfrieren bestimmter Versionen bei gleichzeitigem Parallel\-betrieb von mehren
Versionen durch unterschiedlicher Container. Ein Container kann unter
unschädlichen Betriebssystemen und Plattformen auf gleiche Weise genutzt
(ausgeführt) werden, ohne das Neu\-konfigurationen oder Installationen nötig
sind (abgesehen von Docker selbst).  \\

Im Gegensatz zu anderen Virtualisierungstechniken hat Docker trotz
Virtualisierungstechniken keinen großartigen Performance Verlust weshalb es vor
anderen Visualisierungen wie VirtualBox vorzuziehen ist.  \\

\textbf{Von dem Gebrauch klassischer Virtualisierungstechniken ist aus
    Performance\-gründen dringend abzuraten}. Ein Beispiel ist VirtualBox.


\section{Literatur}%
\label{sec:literatur}

Nachfolgende Quellen sind zum Wissensaufbau oder als Nachschlagewerk zu
empfehlen.

\begin{description}
    \item[Host, IDS und Tool einrichten:] \cite{PhyTec:Development_Guid},
        \cite{Yocto_Eclipse_Plugin} und    \cite{Gonzalez2018:Embedded_Linux_Development_Using_Yocto_Project_Cookbook_2nd}
    \item[Flash, HW Schnittstellen und \acs{DTB}:] \cite[
        \textit{Booting the System} und
        \textit{Updating the System}]{
            Pytec:BSP_Manual}
    \item[Einführung und Beispiele] \cite{Gonzalez2018:Embedded_Linux_Development_Using_Yocto_Project_Cookbook_2nd}
    \item[YouTube Tutorial Serie:] \cite[Live Coding with Yocto
        Project][]{Yocto:YouTube}
    \item[Nachschlagewerk:]\cite{Yocto:Mega_Manual}
    \item[Linux Treiber Entwicklung:] \cite{Quade2015} und \cite{Corbet2005}
    \item[Linux / Unix SystemCall Referenz:] \cite{Kerrisk2010} und \cite{Rago2013}
    \item[Phytec Mira Board Downloads:] \cite{PhyTec:Mira_Downloads}

\end{description}













