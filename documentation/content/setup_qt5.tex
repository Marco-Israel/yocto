
\chapter{QT5 einrichten}%
\label{cha:qt5_einrichten}

\section{Qt Creator einrichten}%
\label{sec:qt_creator_einrichten}


Wie sich QT zur Cross-Entwicklung und Remote debugging einrichten und in
Kombination mit einem bitbake-SDK nutzen lässt ist beschrieben unter

\begin{itemize}
    \item \cite[Seite
        269-276]{Gonzalez2018:Embedded_Linux_Development_Using_Yocto_Project_Cookbook_2nd}
    \item \cite [Working with Qt Creator]{PhyTec:Development_Guid}
\end{itemize}

\section{Arbeiten mit Qt Creator}%
\label{sec:arbeiten_mit_qt_creator}

Der Workflow, wie sie QT-Anwendungen mittels Qt Creator entwickeln und in
Bitbake einbinden lassen, ist beschrieben unter:
    \cite[Seite
        277-285]{Gonzalez2018:Embedded_Linux_Development_Using_Yocto_Project_Cookbook_2nd}


