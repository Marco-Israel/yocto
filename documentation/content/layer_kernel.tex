
\chapter{Kernel Module entwicklung}%
\label{cha:module_entwicklung}


\section{Kernel Module}%
\label{sec:kernel_module}

\subsection{Neues Kernel Modul}%
\label{sub:neues_kernelmodul}


\subsection{Kernel anpassen - features aktivieren}%
\label{sub:kernel_anpssen_features_aktivieren}

Wie unter \cite[Seite 109]{Gonzalez2018:Embedded_Linux_Development_Using_Yocto_Project_Cookbook_2nd}
beschrieben, existieren verschiedene Möglichkeiten ein bereits im Linux Kernel
vorhandenes Modul oder \gls{Feature} zu aktivieren. Z.B.

\begin{itemize}
    \item manuelles Anpassen der Kernel \textit{.config} Datei
    \item GUI oder Menü basiert mittels
        \begin{itemize}
            \item menuconfig
            \item xconfig
            \item gconfig
        \end{itemize}
    \item \textit{bitbake -c menuconfig virtual/kernel}
\end{itemize}

Seite \cite[Seite 114 und Seite 118]{Gonzalez2018:Embedded_Linux_Development_Using_Yocto_Project_Cookbook_2nd}
zeigt letztes beispielhaft und beschreibt wie \textbf{Änderungen dauerhaft
    gespeichert werden können}


\section{Software Module}%
\label{sec:software_module}

Neue, selbst erstellte Kernel Module lassen sich durch die nachfolgenden
Konfigurations\-variablen zum Image \textit{rootfs} hinzufügen.

Die Variablen können beispielhaft in einem Image-Recipe oder in einer der
Konfigurationsdateien wie \textit{./conf/local.conf} oder
\textit{machine/<yourMachine.conf>}  definiert werden.

\begin{itemize}
    \item MACHINE\_ESSENTIAL\_EXTRA\_RDEPENDS\_append +=  <module-recipe>
    \item MACHINE\_ESSENTIAL\_EXTRA\_RRECOMMENDS\_append += <module-recipe>
    \item MACHINE\_EXTRA\_RDEPENDS\_append += <module-recipe>
    \item MACHINE\_EXTRA\_RRECOMMENDS\_append += <module-recipe>
\end{itemize}

Um das Modul \textit{automatisch, beim booten des Kernels} zu laden, muss
zusätzlich definiert werden:

\begin{itemize}
    \item KERNEL\_MODULE\_AUTOLOAD\_append = \glqq< ><kernel-Modul> \grqq
    \item KERNEL\_MODULE\_AUTOLOAD\_append += <kernel-Modul>
    \item KERNEL\_MODULE\_AUTOLOAD += <kernel-Modul>
\end{itemize}


\subsection{Eigene Kernel modul erstellen und einbinden}%
\label{sub:eigene_kernel_modul_erstellen_und_einbinden}


Seite \cite[121-125]{Gonzalez2018:Embedded_Linux_Development_Using_Yocto_Project_Cookbook_2nd}
zeigt beipielhaft, wie sich ein eigene Kernelmodule erstellen und in die
Buildumgebung Bitbake einminden lraässt. Der \textit{poky} meta-layer beinhaltet
zudem ein beispielhaftes \textit{hello-world} Kernel Modul, mitsamt
beispielhafter \textit{Makefile} und \textit{recipe.bb} Datei.
\\
Ein umfangreiches Handbuch zur Linux Kernel und Modulentwicklung stellen die
beiden Bücher dar:
\begin{itemize}
    \item \cite{Quade2015}
    \item \cite{Corbet2005}
\end{itemize}

Nachschlagewerke zum Thema Linux/Unix Kernel Schnittstellen (System-Calls) sind
die Bücher:
\begin{itemize}
    \item \cite{Kerrisk2010}
    \item \cite{Rago2013}
\end{itemize}
