
\chapter{Module entwicklung}%
\label{cha:module_entwicklung}


\section{Kernel Module}%
\label{sec:kernel_module}

\subsection{Neues Kernelmodul}%
\label{sub:neues_kernelmodul}


\subsection{Kernel anpssen - features aktivieren}%
\label{sub:kernel_anpssen_features_aktivieren}


\section{Software Module}%
\label{sec:software_module}


\section{Flashen  des image}%
\label{sec:flashen_des_image}

Es existieren verschiedene Wege einen Linux Kernel über das Netzwerk zu booten
oder in den Speicher zu schreiben.

\begin{itemize}
    \item Netzwerk
    \item sdcard
    \item usb
    \item spi
    \item serial
    \item weitere
\end{itemize}
















%
%to add the modules into your kernel image root fs, you need to define one of the
%following variables in your machine/<yourMachine.conf> configuration file
%
%MACHINE_ESSENTIAL_EXTRA_RDEPENDS
%
%MACHINE_ESSENTIAL_EXTRA_RRECOMMENDS
%
%MACHINE_EXTRA_RDEPENDS
%
%ACHINE_EXTRA_RRECOMMENDS
%
%
%If you like to load the module during startup, you also need to define the
%following variable in your machine/<yourMachine.conf> file.
%
%KERNEL_MODULE_AUTOLOAD
%
%For further information please grep the yocto mega manual in "Variables Glossary"
%
