\chapter{Source code development und debugging}%
\label{cha:source_code_development}


\section{Cross development}%
\label{sec:cross_development}
Es existieren verschiedene Wege um Anwendungen für Ziel\-plattformen zu
entwickeln. Yocto setzt dabei auf die beiden nachfolgenden; zum einen die
Bereitstellung einer gekapselten Entwicklungsumgebung in Form eines \glspl{SDK},
zum anderen die Entwicklung unter Zuhilfenahme der Build\-umgebung Bitbake
mitsamt seiner Tools.  \\

Nachfolgend beide Wege im Überblick:

\subsection{Entwicklung mit dem Yocto SDK}%
\label{sub:entwicklung_mit_dem_yocto_sdk}
Dieser Weg dient der normalen Entwicklung von Software\-komponenten. Das Yocto
SDK liefert eine gekapselte Shell-Umgebung mit vordefinierten
Umgebung\-variablen sowie allen nötigen Header Dateien, Libraries usw. welche
auf der Ziel\-plattform vorhanden sein werden. Hierzu bildes es die
Ordner\-struktur (das s.g. \gls{rootfs}) der Ziel\-plattform in einem Ordner
ab. \\


\begin{description}
    \item[1. Shell Umgebungsvariablen laden] Um die SDK Umgebung zu nutzen ist
        es nötig, eine Konfigurationsdatei in die Shell zu laden.

\begin{lstlisting}[frame=single,language=bash,caption={Einrichten der SDK
        Umgebung}]
[user@host]/yoctopath $: source ./enviroment-setup-*.sh
\end{lstlisting}


    \item[2. Eclipse starten] Abhängig vom Anwendung\-fall, anschließend in der
        selben Shell Eclipse oder QTCreator starten.

\begin{lstlisting}[frame=single,language=bash,caption={Start von Eclipse in der
        zuvor konfiguierten Umgebung}]
[user@host]/yoctopath $: eclipse &
\end{lstlisting}

\end{description}



\subsection{Entwicklung mit Yocto Tools}%
\label{sub:entwicklung_mit_yocto_tools}
Diese Software Entwicklungs\-art dient mehr dazu Änderungen an existierendem
Sourcecode durchzuführen. Hierbei stehen zwei tools zur Verfügung:

\begin{description}
    \item[devtool <cmd> ] Zum einen steht das Yocto-Tool \textbf{devtool} mit
        verschiedene Kommandos bereit. Eine Übersicht über Kommandos und
        Anwendungs\-beispiele liefert die Yocto Web\-seite oder \textit{devtool
            --help}.
    \item[bibake -c devshell <recipie>] Erzeugt bzw. öffnet eine
        vorkonfigurierte Shell ensprechend den Metadaten eines \textit{recipie}
        zu einer Source Datei.
\end{description}



\section{Remote Debugging}%
\label{sec:remote_debugging}



