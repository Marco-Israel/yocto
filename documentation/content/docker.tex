\chapter{Arbeiten mit Docker}%
\label{cha:arbeiten_mit_docker}

Wie bereits einleitend  beschrieben, sind für das
Arbeiten mit Bitbake, mit samt seiner Zusatztools, verschiedene Linux/Unix
Standard Tools in zu Bitbake kompatiblen Versionen notwendig. Docker wird als
Virtualisierungsumgebung genutzt, um eine lauffähige Bitbake build Umgebung
bereit zu stellen ohne dabei großartig preformance durch
Virtualisierungstechniken zu verlieren, wie es bei alternativen der Fall ist.
siehe hierzu auch \fullref{sec:einleitung_docker}.
\\
Ein solches Dockerimage liegt bereits vor und muss lediglich \textit{gebaut} und
kann anschließend genutzt werden. Siehe hierzu \ref{dockerimage_bauen} und
\ref{sec:Dockerimage_ausführen}


\section{Dockerimage_bauen}%
\label{sec:dockerimage_bauen}
Um ein Docker image zu bauen, die nachfolgenden Schritte ausführen
\begin{itemize}
            \item Docker Service starten
            \item Docker-yocto Image bauen:
                \textbf{ \textit{docker build -t Dockerfile} }.
            \item Hilfe liefert docker --help oder die \textbf{Internetseite}
            \item Image starten durch ausführen von \glqq ./run.sh bash \grqq
\end{itemize}

Es ist wichtig, das das \textbf{Dockerfile} im aktuellen Arbeitsverzeichnis
liegt \textbf{und den Namen \textit{Dockerfile} trägt}.

\section{Dockerimage_ausführen}%
\label{sec:dockerimage_ausfuhren}
Um das Dockerimage zu starten wurde der Aufruf in das Bashscript \textit{run.sh}
gekapselt.
\\
Das \textit{run.sh} script erzeugt zunächst ein \textit{dockerjobs.sh} Script
oder benötigt ein bereits vorhandenes. Das \textit{dockerjobs.sh} skript
enthält die Befehle, welche durch \textit{bitbake} (in einem batch Betrieb}
ausgeführt werden sollen. \textbf{Das Script kann händisch geändert oder um
weitere Aufgaben ergänzt werden.}

Das run.sh script startet
den Docker container in definierter Version (gesetzt über Parameter oder direkt
innerhalb des run.sh scripts). Das \textit{dockerjobs.sh} wird anschließend innerhalb Docker durch das
Image aufgerufen und enthält alle Aufgaben welche durch den Container in einem
Batchmodus erfüllt werden sollen.
\textbf{Das Script kann händisch geändert oder um weitere Aufgaben ergänzt
werden}. Es wird initial erzeugt und \textbf{überschrieben} wenn \textit{run.sh}
mit Parametern aufgerufen wird.

\subsection{\textit{run.sh} Ausführungsoptionen}%
\label{sub:textit_run_sh}
\begin{description}
    \item[\textit{run.sh} ohne Parameter und Kommando] startet den Docker Container und führ
        das \texit{dockerjobs.sh} Script aus,  insofern es vorhanden ist.
    \item[\textit{run.sh} <cmd-to-execute>] \textbf{Überschriebt} das
        dockerjobs.sh script, startet den Container und führt das neue
        \textit{dockerjobs.sh} script mitsamt des übergebenen Parameters aus
    \item[\textit{run.sh bash}] Startet den Dockercontainer im
        den Dockercontainer selbst in einen \textit{interactiven modus}.
        Anschließend kann in dem Container bzw. in der neuen Shell, gearbeitet
        werden, als wenn man sich auf dem lokalen host befände mitsamt aller
        nötigen Bitbake tools.
    \item[\textit{run.s} -i <image-name>] Image oder Projektname unter dem
        gearbeitet werden soll. Der default lässt sich in dem \textit{run.sh}
        Skript definieren.
    \item[\textit{run.sh} -t <tag-version>] Die Version, genauer der Tag, des
        Dockerimages welcher gestartet werden soll. Der default lässt sich in dem \textit{run.sh} Skript definieren.
\end{description}

\subsection{Dockerscript einstellen}%
\label{sub:dockerscript_einstellen}
\begin{itemize}
    \item In dem \textit{run.sh} Skript lässt sich die default Tag-Version des
        Dockerimages einstellen.
    \item In dem \textit{run.sh} Skript lässt sich der default zu verwendende
        Projektname bzw. das zu verwendende Projektverzeichnis definieren.
    \item Das \textit{dockerjobs.sh} kann manuell erstellt, geändert oder um
        weitere Batchmode-Befehle erweitert werden. Je nach Aufrufoptionen wird
    das \textit{dockerjobs.sh} Skript jedoch überschrieben.
\end{itemize}






