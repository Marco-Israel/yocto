\chapter{Setup your host}%
\label{cha:setup_host}


\section{Erforderliche Pakete}%
\label{sec:erforderliche_pakete}

Das yocto Docker file zeigt eine Liste aller nötigen Ubuntu Pakte die zum
Arbeiten mit der Yocto / OpenEmbedded Build Umgebung auf einem  Host benötigt
werden, sollte nicht mit dem Docker Image gearbeitet werden wollen. \\


\textbf{Zusätzlich} sind die nachfolgenden Pakete werden zum Arbeiten auf dem lokalen
Host benötigt. Nähere Informationen zu den Paketen, Quellen, Konfigurationen
usw. sind auf verschiedenen Internetseiten zu finden.

\begin{itemize}
    \item git
    \item docker
    \item TFTP Server
    \item NFS Server
    \item microcom
    \item eclipse
    \item qt5
    \item qt5Creator
    \item openssh-server
\end{itemize}

\section{Host Konfiguration}%
\label{sec:host_konfiguration}

Nachfolgende Pakete benötigen weitere Konfigurationen
\begin{description}
    \item[Docker: ]
        \begin{itemize}
            \item[ ]
            \item Docker Service starten
            \item Docker-yocto Image bauen:
                \glqq docker build -t yocto . \grqq
            \item Hilfe liefert docker --help oder die Internetseite
            \item Image starten durch ausführen von \glqq ./run.sh bash \grqq
        \end{itemize}

    \item[TFTP Server:]
        \begin{itemize}
            \item[ ]
            \item TFTP Austausch Ordner anlegen und Zugriffsrechte definieren
            \item stat-alone deamon (/etc/default/tftpd-hpa) oder xinitd Service
                (/etc/xintd.d/tftp) konfigurieren
            \item Server neu starten
            \item \textbf{BEISPIEL} im \glqq BSP Manual\grqq unter  phytec.de;
                    Stichwort \glqq Booting the Kernel from Network\grqq
                (Booting\_the\_Kernel\_from\_Network) \cite{Pytec:BSP_Manual}
                oder unter \cite[S.
                44]{Gonzalez2018:Embedded_Linux_Development_Using_Yocto_Project_Cookbook_2nd}
        \end{itemize}

    \item[NFS Server:]
        \begin{itemize}
            \item[ ]
            \item NFS Server konfigurieren (/etc/exports)
            \item NFS Server neu starten
            \item \textbf{BEISPIEL} im \glqq BSP Manual\grqq unter  phytec.de;
                    Stichwort \glqq Booting the Kernel from Network\grqq
                (Booting\_the\_Kernel\_from\_Network) \cite{Pytec:BSP_Manual}
                oder unter \cite[S.
                45]{Gonzalez2018:Embedded_Linux_Development_Using_Yocto_Project_Cookbook_2nd}
        \end{itemize}
    \item[Microcom]
        \begin{itemize}
            \item[ ]
            \item Der Parameter --port Definiert die Serielle Schnittstelle.
            \item Weiteres ist unter Manual Seite zu finden.
        \end{itemize}
    \item[Eclipe]
        \begin{itemize}
            \item Weiters im Kapitel \ref{cha:setup_eclipse}; Seite
                \pageref{cha:setup_eclipse}
        \end{itemize}
    \item[QT5Creator]
        \begin{itemize}
            \item[ ]
            \item Weiters im Kapitel \ref{cha:setup_qtcreator}; Seite
                \pageref{cha:setup_qtcreator}
        \end{itemize}
    \item[Yocto Areitsverzeichnis]
        \begin{itemize}
            \item[ ]
            \item Erstellen eines globalen Arbeitsverzeichnisses.  Z.B.
                /opt/yocto \item Setzten der Rechte rwx Rechte für alle user
                (\glqq others\grqq).
        \end{itemize}
    \item[Python2 als standart interpreter] Yocto/Openembedded Tools bauen auf
        Python2 auf. Daher ist es nötig einen  symlink  oder alias
        auf Python2 zu setzten. Z.b. \textit{alias python=python2}
    \item[Proxy und Routen]
    Je nach Netzwerkinfrastruktur müssen Proxy und Netzwerkrouten auf
    dem lokalen Host gesetzt werden. Beispielsweise für die Tools:
        \begin{itemize}
            \item Git
            \item wget
            \item apt-get
            \item https\_proxy und http\_proxy
        \end{itemize}
        \textbf{Informationen} hierzu liefert die das Yocto Manual oder das
            Yocto Wiki unter dem Stichwort \glqq \textbf{Working Working Behind
            a Network Proxy} \grqq (Working\_Behind\_a\_Network\_Proxy)

\end{description}

