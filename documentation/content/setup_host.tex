\chapter{Einrichtung des Host}%
\label{cha:setup_host}


\section{Erforderliche Pakete}%
\label{sec:erforderliche_pakete}

Das \textit{\gls{Dockerfile}} \glqq Yocto\grqq zeigt eine Liste aller nötigen
Ubuntu Pakte die zum Arbeiten mit der Yocto / OpenEmbedded Build Umgebung auf
einem Host benötigt werden. Beim Arbeiten mit dem Docker Image sind alle nötigen
Tools bereits nach dem erstellen des Docker Container in dem Container in
kompatiblen Versionen  enthalten.  \textit{Bitbake} und andere Tools können
direkt genutzt werden.\\ Der Container lässt sich bauen mit \textit{docker build
-t Dockerfile}.

\textbf{Zusätzlich} sind die nachfolgenden Pakete zum Arbeiten auf dem lokalen
Host nötig oder empfehlenswert. Nähere Informationen zu den Paketen, Quellen,
Konfigurationen usw. sind auf verschiedenen Internetseiten zu finden.

\begin{description}
    \item [\gls{GIT}:] Manuels Herunterladen von Metadaten oder anderen GIT
        Repositories
    \item [Docker:] Zwingend erforderlich um einen Docker Container bauen und
        ausführen zu können.
    \item [\gls{TFTP} Server] Bereitstellen des fertigen Liunx Images, Root File
        Systems sowie des \acl{DTB}
    \item [\gls{NFS} Server] Bereitstellen eines globalen Dateisystems über das
        Netzwerk
    \item [microcom] Werkzeug zum verbinden mit einer serielle Schnittstelle
    \item [eclipse] Entwicklungs-IDE mit Unterstützung zum Cross-Kompilieren als
        auch zum Remote debuggen
    \item [QT5 und QT5 Creator] (Cross-)Entwicklung und Remote debugging IDE für
        grafische Oberflächen in C++ als \acs{WYSIWYG} Editor.
    \item [Openssh-Server] Verbinden und Übertragen von Dateien mit/zu dem
        eingebetteten Linux auf der Zielhardware.
\end{description}

\section{Host Konfiguration}%
\label{sec:host_konfiguration}

Nachfolgende Pakete benötigen weitere Konfigurationen
\begin{description}
    \item[Docker: ]
        \begin{itemize}
            \item[ ]
            \item Docker Service starten
            \item Docker-yocto Image bauen:
                \textbf{ \textit{docker build -t Dockerfile} }.
            \item Hilfe liefert docker --help oder die \textbf{Internetseite}
            \item Image starten durch ausführen von \glqq ./run.sh bash \grqq
        \end{itemize}

    \item[TFTP Server:]
        \begin{itemize}
            \item[ ]
            \item TFTP Austausch Ordner anlegen und Zugriffsrechte definieren
            \item stat-alone deamon (/etc/default/tftpd-hpa) oder xinitd Service
                (/etc/xintd.d/tftp) konfigurieren
            \item Server neu starten
            \item \textbf{BEISPIEL} im \glqq BSP Manual\grqq unter  phytec.de;
                    Stichwort \glqq Booting the Kernel from Network\grqq
                (Booting\_the\_Kernel\_from\_Network) \cite{Pytec:BSP_Manual}
                oder unter \cite[Seite
                44]{Gonzalez2018:Embedded_Linux_Development_Using_Yocto_Project_Cookbook_2nd}
        \end{itemize}

    \item[NFS Server:]
        \begin{itemize}
            \item[ ]
            \item NFS Server konfigurieren (/etc/exports)
            \item NFS Server neu starten
            \item \textbf{BEISPIEL} im \glqq BSP Manual\grqq unter  phytec.de;
                    Stichwort \glqq Booting the Kernel from Network\grqq
                (Booting the Kernel from Network) \cite{Pytec:BSP_Manual}
                oder unter \cite[Seite
                45]{Gonzalez2018:Embedded_Linux_Development_Using_Yocto_Project_Cookbook_2nd}
        \end{itemize}
    \item[Microcom]
        \begin{itemize}
            \item[ ]
            \item Der Parameter --port Definiert die Serielle Schnittstelle.
            \item Weiteres ist unter Manual Seite zu finden.
        \end{itemize}
    \item[Eclipe]
        \begin{itemize}
            \item Weiters im Kapitel \fullref{cha:setup_eclipse}
        \end{itemize}
    \item[QT5Creator]
        \begin{itemize}
            \item[ ]
            \item Weiters im Kapitel \fullref{cha:qt5_einrichten}
        \end{itemize}
    \item[Yocto Areitsverzeichnis]
        \begin{itemize}
            \item[ ]
            \item Erstellen eines globalen Arbeitsverzeichnisses.  Z.B.
                /opt/yocto \item Setzten der Rechte \textbf{ \textit{rwx}}
                Rechte für alle user (\glqq others\grqq).
        \end{itemize}
    \item[Python Version als standart interpreter] Yocto/Openembedded Tools bauen auf
        je nach Version auf Python Version 2 oder 3 auf. Daher ist es nötig einen
        Symlink  oder Alias auf die jeweilige Python Version zu setzten.
        Z.b. \textit{alias python=python2}
    \item[Proxy und Routen]
    Je nach Netzwerkinfrastruktur müssen Proxy und Netzwerkrouten auf
    dem lokalen Host gesetzt werden. Beispielsweise für die Tools:
        \begin{itemize}
            \item GIT
            \item wget
            \item apt-get
            \item https\_proxy und http\_proxy
        \end{itemize}
        \textbf{Informationen} hierzu liefert die das Yocto Manual oder das
            Yocto Wiki unter dem Stichwort \glqq \textbf{Working Working Behind
            a Network Proxy} \grqq (Working Behind a Network Proxy)

\end{description}

