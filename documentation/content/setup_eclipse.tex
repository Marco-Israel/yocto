\chapter{Setup Eclipse}%
\label{cha:setup_eclipse}

\section{Installation,  Einrichtung und Plugins}%
\label{sec:installation__einrichtung_plugins}

Zur Entwicklung von C/C++ kann Eclipse CDT verwendet werden. \\


Zudem sind zum Cross compilieren und Remote Debuggen sowie zum remote
Deployen nachfolgende Plugins bzw zusätzliche Eclipse-Softwaremodule nötig:
(\textit{Help} -> \textit{Install new Software} )

\begin{itemize}
    \item C/C++ Remote (Over TCF/TE) Run/Debug Launcher.
    \item Remote System Explorer User Actions
    \item TM Terminal via Remote System Explorer
    \item TCF Target Explorer
\end{itemize}

Des weiteren stellt die Yocto download Seite \textit{downloads.yoctoproject.org}
ein SDK Plugin bereit, welches das Konfigurieren von Remote Einstellungen
für jeweilige Entwicklungsprojekt vereinfacht und zusammenfasst. Zum
installieren muss beispielsweise zu den Installations-Quellen in eclipse
\textbf{http://downloads.yoctoproject.org/releases/eclipse-plugin/2.6.1/oxygen/}
hinzu gefügt werden.


\section{Cross Compile und Remote Debugging Einstellungen}%
\label{sec:cross_compile_und_remote_debugging_einstellungen}

Eine Anleitun mit Beispielen zum Cross Compile und Remote Debugging unter
eclipse liefern:
\begin{itemize}
    \item \textbf{\cite[Working with Eclipse]{PhyTec:Development_Guid}} Anleitung wie o.g. ohne Plugin
        unter Verwendung eines SDKs in eclipse eingerichtet werden kann
    \item \textbf{\cite[Seite
        249-269]{Gonzalez2018:Embedded_Linux_Development_Using_Yocto_Project_Cookbook_2nd}}
        Anleitung wie das Yocto Plugin für o.g. verwendet werden kann.
    \item \textbf{\cite{Variwiki:Yocto_Eclipse_Plugin}} Wie oben, zeigt die Verwendung des Yocto Plugins.

\end{itemize}
