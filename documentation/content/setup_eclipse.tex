\chapter{Setup Eclipse}%
\label{cha:setup_eclipse}

\section{Installation,  Einrichtung und Plugins}%
\label{sec:installation__einrichtung_plugins}

Zur Entwicklung von C/C++ kann Eclipse CDT verwendet werden. \\


Zudem sind zum Cross compilieren und Remote Debuggen sowie zum remote
Deployen nachfolgende Plugins bzw zusätzliche Eclipse-Softwaremodule nötig:
(\textit{Help} -> \textit{Install new Software} )

\begin{itemize}
    \item C/C++ Remote (Over TCF/TE) Run/Debug Launcher.
    \item Remote System Explorer User Actions
    \item TM Terminal via Remote System Explorer
    \item TCF Target Explorer
\end{itemize}

Des weiteren stellt die Yocto download Seite downloads.yoctoproject.org
ein SDK Plugin bereit, welches das Konfigurieren von Remote Einstellungen
für jeweilige Entwicklungsprojekt vereinfacht und zusammenfasst. Zum
installieren muss beispielsweis zu den Installationsquellen in eclipse
\textbf{http://downloads.yoctoproject.org/releases/eclipse-plugin/2.6.1/oxygen/}
hinzu gefügt werden.


\section{Cross Compile und Remote Debugging Einstellungen}%
\label{sec:cross_compile_und_remote_debugging_einstellungen}

TODO
