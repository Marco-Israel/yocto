\chapter{SKD bereitstelle}%

Die Aufgaben eines \ac{SDK} sind beschrieben in \gls{SDK}
\label{cha:skd_bereitstelle}


\section{erzeugen eines SDKs}%
\label{sec:erzeugen_eines_sdks}

Es existieren verschiedene Wege ein Projektbezogenes SDK zu erzeugen.


\begin{itemize}
    \item Ausführen eines der toolchain recipes. Z.B.:
        \begin{description}
            \item [standard SDK] bitbake meta-toolchain
            \item [erweitertes SDK] bitbake meta-extsdk-toolchain
            \item [sdk mit integriertem QT5 support] bitbake meta-toolchain-qt5
            \item [\ldots] Weitere SKDs
        \end{description}

    \item Ausführen des SDK tasks eines \textit{image recipes}. Z.B.:
        \begin{description}
            \item [standard SDK] bitbake -c populate\_sdk <image-recipes>
            \item [erweitertes SDK] bitbake -c populate\_sdk\_ext
                <image-recipes>
            \item [sdk mit integriertem QT5 support] bitbake -c
                populate\_sdk\_qt5
            \item [\ldots] Weitere SKDs
        \end{description}
\end{itemize}

Weitere toolchain recipes sind zu finden unter \cite{OEI} recipes
Stich\-wortsuche \textit{toolchain}.


\section{Nutzen eines SDKs}%
\label{sec:nutzen_eines_sdks}

Um ein erzugtes SDK zu nutzen:
\begin{enumeration}
    \item SDK-script z.B. aus \textit{deploy/sdk/\ldots} ausführen und das SDK
        so entpacken.
    \item Shell configurieren durch \textit{sourcen} des enviroment Skriptes. Zu
        finden innerhalb des zuvor etpackten SDKs.
    \item Testen z.b. durch nachvollziehen ob umgebungsvariablen und Pfade
        richtig gesetzt wurden:
        \begin{itemize}
            \item echo $PATH
            \item echo $CC
        \end{itemize}
\end{enumerate}


weiteres zum arbeiten mit einem SDK ist zu finden in \cite{Gonzalez2018} oder
\cite{PTDG}.
