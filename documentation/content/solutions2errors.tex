\chapter{Lösung für bekannte Fehler}%
\label{cha:losung_fur_bekannte_fehler}


\section{\textbf{ERROR:} Fehler beim Bauen eines recipes>}
\label{sec:fehler_beim_bauen_eines_images}


\subsection{Lösung}%
\label{sub:losung_fehler_beim_bauen_eines_images}


\begin{enumerate}
    \item Ausführen von \textit{bitbake -c cleanall <recipes>}
    \item \textit {Force} kompilieren mit  \textit{bitbake -f <recipes>}
    \item Sollte nichts helfen, je nach situation einen oder alle der
        nachfolgenden Ordner löschen.
    \begin{itemize}
        \item \textit{sstate} Ordner
        \item \textit{deploy} Ordner
        \item \textit{tmp} Ordner
        \item Alles weitere im Projekt Ordner, mit Ausnahme des \textit{conf}
            Ordners.
    \end{itemize}
\end{enumerate}



\section{Logging in Recipes}%
\label{sec:logging}
In Bitbake recipes lässt sich \textbf{python} oder \textbf{bash} \textbf{loggig}
nutzen. Dabei werden durch die Yocto und und OpenEbedded Community bereits
vordefinierte \textit{Logging Classes} (*.bbclass) bereit gestellt die
unterschiedliche Informationen loggen können, angefangen von:
\begin{itemize}
    \item \textit{plain} und \textit{notes}
    \item über \textit{warnings} und \textit{(fatal-) errors}
    \item bis bin zu \textit{debug} informationen
\end{itemize}
 Eine Übersicht ist gelistet unter
 \cite[S.79-80]{Gonzalez2018:Embedded_Linux_Development_Using_Yocto_Project_Cookbook_2nd}.

% gefolgt von Anwendungsbeispielen unter \cite[S.80-82]{Gonzalez2018:Embedded_Linux_Development_Using_Yocto_Project_Cookbook_2nd}.


\section{Abhängigkeiten (Dependencies) visualisieren }%
\label{sec:abhangigkeiten_dependencies_visualisieren_}
Bitbake ist in der Lage, Abhängigkeiten zwischen Recipes bzw. zwischen Software
Componenten als Graphen zu visualisieren. Hierbei wird mittels \textit{Graphvz}
{.dot} files zwischen den Abhängigkeiten generiert. \textit{*.dot} Dateien
lassen sich anschließend in eine Grafik oder PDF umwandeln. Beispielhaft:


\begin{lstlisting}[frame=single,language=bash,caption={Dependency Graph}]
[user@host]/yoctopath $: source ./[<path>]enviroment-setup-*.sh
[user@host]/yoctopath $: bitbake -g <recipes> [-I <dependencie to ignore> ]
[user@host]/yoctopath $: dot -Tpng *.dot -o dependancy.png
[user@host]/yoctopath $: dot -Tpdf *.dot -o dependancy.pdf.
\end{lstlisting}

Alternativ stellt Yocto/Openembedded einen \textit{dependency explorer} bereit.
Hierzu Bitbake aufrufen mit
\begin{itemize}
    \item bitbake -g \textbf{-u taskexp} <recipe>
\end{itemize}

Weitere Abhängikeits-Probleme zwischen Dateien/Versionen lassen sich ermitteln
über die Dateien:
\begin{itemize}
    \item ./tmp/stamps/sigdata
    \item ./tmp/stamps/siginfo
\end{itemize}

Hierbei handelt es sich um \textit{Python-Datenbanken} die sich wie folgt
ausgeben und mit dem aktuellen Abseitsstand vergleichen lassen:

\begin{itemize}
    \item bitbake-dumpsig [sigdata / siginfo]
    \item bitbake-diffsig [sigdata / siginfo]
\end{itemize}




