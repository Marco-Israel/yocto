
\chapter{Bekannte Fehler - Pittfalls}%
\label{cha:bekannte_fehler}


\section{Never do}%
\label{sec:Never_do}

Nachfolgende actionen füren in jedem Fall zu einem Fehler

\begin{description}
    \item[Unterstrich \glqq \_ \grqq im Namen]  Vergebe niemals einen Namen mit
        einem Unterstrich ( \_ ); z.B. für ein Recipe oder einen Ordner
    \item[kein CamelCase] Namen für Recipies oder beispielsweise Ordner müssen
        klein geschreiben werden. CamelCase oder Großschreibung ist nicht
        zulässig.
    \item[Umbenennung von Dateien oder Pfaden] Benenne niemals einen Ordner,
        einen Pfad oder ähnliches um. Anderfalls ist ein komletter Neuanfang mit
        leerer \textit{Shell} und leerem \textit{temp}
        und \textit{sstate\_cache} Ordner nötig.
    \item Updaten, austauschen, \ldots von meta-layern, paketen, Konfigurationen
        \ldots. Neue Abhängigkeiten und veränderungen führen oft zu Fehlern.
\end{description}


\section{Must do}%
\label{sec:must_do}

\begin{itemize}
    \item Es \textbf{muss} der \textit{sstate chache} manuell gelöscht werden
        wenn:
        \begin{itemize}
            \item Kernel Konfigurationen verändert wurden. Z.B. durch
                \textit{menuconfig} oder \textit{bitbake -c configure virtual/kernel}
            \item Module zum Kernel-Autoload hinzu gefügt wurden. Siehe hierzu
                \fullref{sec:software_module}
            \item Eine \textbf{Maschinenkonfiguration}
                \textit{machine/mymachine.conf} verändert wurde.
            \item Eine \textbf{Distributionsconfiguration}
                \textit{distro/mydistro.conf} verändert wurde.
        \end{itemize}
    \item Sollte alles nichts helfen, muss auch der \textit{tmp} Ordner gelöscht
        werden oder ein \textit{bitbake --force <command\_and\_recipe}
        aufgerufen werden. Siehe herzu auch \fullref{sub:loesungsschritte}
\end{itemize}

\section{Generelle Empfehlungen}%
\label{sec:generelle_empfehlungen}

Beim Arbeiten mit Bitbake ist zu empfehlen:

\begin{itemize}
    \item Kleine Änderungen durchführen und testen.
    \item Eigene Meta-dateien und Änderungen in bestehenden Konfigurationen und
        Metadateien in einem Reposetorie Versionieren. In einem Fehlerfall ist
        so ein Rollback möglich.
    \item Sichere regelmässig local die Ordner
       \begin{itemize}
           \item \textit{sstate-cache}
           \item \textit{tmp}
           \item \textit{deploy} bzw. \textit{tmp/deploy}
       \end{itemize}
    \item Halte immer ein Lauffähiges \textit{Image}, \textit{rootfs},
    und einen funktionierenden
    \textit{device tree blob} und ggf. ein funktionierendes \textit{SDK}
    bereit; u.a. um  Fehler der Hardware ausschließen zu können.
    \item Vorgehensweise beim erstellen einer neuen Konfiguration:
        \begin{itemize}
            \item Eine Bestehende Image / Distro / Machine Configuration
                nutzen und nach eigenen Anforderungen abändern. Änderungen
                jeweils in kleinen schritten durchführen.
            \item Auf einer minimalen poky Konfiguration neu aufbauen. Ggf. an
                anderen Beispielen orientieren.
            \item Eine Bestehende größere Konfigurationsstruktur zu erweitern
                ist \textit{nicht empfehlenswert} da Fehleranfällig und schwer
                zu verstehen.
            \item Must-dos \fullref{sec:must_do} beachten.
            \item Never-do \fullref{sec:never_do} beachten.
        \end{itemize}
    \item Variablen sollten wie folgt erweitert werden:
        \begin{description}
            \item[In Konfigurationsdateien] durch die Operatoren \textit{\_append}
                oder \textit{\_prepend}.
            \item[In Recipes] durch Operatoren \textit{+=}
                oder \textit{.=}.
            \item[Weitere Informationen und Hintergründe unter] \cite[Seite
                160]{Gonzalez2018:Embedded_Linux_Development_Using_Yocto_Project_Cookbook_2nd}
        \end{description}
\end{itemize}



