

\chapter{Grundlage}%
\label{cha:grundlagen}

\section{Bitbake }%
\label{sec:bitbake_buildprozess}



Die Buildumgebung \textit{Bitbake}, im Kern bestehend aus Phython- und
Shellscripen die mit unterschiedlichen Softwareentwicklungstools wie GIT, Make
oder Autotools interagieren, führt in einer geregelten Reihenfolge verschiedene
Aufgaben aus. Die Abbildung auf \cite[S. 20]{Gonzalez2018:Embedded_Linux_Development_Using_Yocto_Project_Cookbook_2nd}
zeigt abstract den Build\-process:

\begin{itemize}
    \item Reihenfolge der angepassten config files gepasst werden.
    \item Parsen der config files.
    \item Buildschritte (Siehe \ref{sec:bitbake_build_tasks} Seite
        \pageref{sec:bitbake_build_tasks})
    \item Packetieren in unterschiedliche Pakettypen
    \item Generieren und bereitstellen von \aclp{SDK}
    \item Aufräum- und Nacharbeiten
\end{itemize}

\section{Bitbake Buildprozess}%
\label{sec:bitbake_build_tasks}

Jedes \textit{recipe} erbt eine Reihe von standard Build-Tasks.  Hierzu
gehören u.a.
\begin{itemize}
    \item Herunterladen (fetchen) und Sammeln von Source Dateien
        (z.B. Git Repositorien, ftp Servern oder lokalen Dateien.)
    \item Übersetzten, konfigurieren, patchen, installieren, verifizieren usw.
        von Paketen auf mehreren Prozessor Kernen
    \item Bereitstellen von Entwicklung und Verwaltungstools
\end{itemize}

Eine Auflistung der Standard-Task kann, mitsamt kurzer Beschreibung,
entnommen werden
\cite[S. 171-172]{Gonzalez2018:Embedded_Linux_Development_Using_Yocto_Project_Cookbook_2nd}

\subsection{Fatch Task}%
\label{sub:fatch_task}
Eine der Hauptaufgaben von Bitbake besteht dadrinne, die jeweiligen
Softwarepakte in Ihren benötigten Versionen und Revisionsständen zusammen aus
unterschidlichen Quellen zusammen zu sammeln und dem Buildsystem zur Verfügung
zu stellen. Solche Quellen können sein:

\begin{itemize}
    \item Lokaler Bitbake-Download ordner/cache. Er enthält bereits einmal
        heruntergeladene Softwarepakete in jeweiligen Revisionständen.
    \item Lokaler Pfad zu Quelldateien; z.b. in einem eclipse oder QT Workspace
    \item Lokale Repository (z.B. lokales GIT Repository)
    \item Netzwerkpfad zu einem Client oder Server im lokalen Netzwerk. Z.B.
        über Freigaben oder einem lokalen FTP Server
    \item Online Repository oder TFTP Server.
    \item Alternatives lokales oder Remote (online) Repository.
\end{itemize}

Die Reihe\-folge in welcher nach Source-Dateien gesucht werden soll ist zum
einen definiert durch:

\begin{itemize}
    \item Das Recipe selbst welches das Software\-pakete innerhalb Bitbake bauen
        soll
    \item durch Konfigurationsdateien wie \textit{./conf/local.conf}
    \item sowie durch eine allgemein fest vorgegeben Reihefolge innerhalb
    bitbakes. Siehe
    \cite[S.53]{Gonzalez2018:Embedded_Linux_Development_Using_Yocto_Project_Cookbook_2nd}
\end{itemize}

\section{Configurations-Dateien *.conf}%
\label{sec:configurations_dateien_conf}
Bitbake, sowie die Buildprozesse der einzelnen Recipes
werden gesteuert durch unterscheidliche \textbf{recipe-lokale} und
\textbf{globale} configurationsdateien.



\section{\textit{Operatoren} zum Verändern von Variablen}
\label{sec:grundlagen_operatoren}

Nachfolgende operatoren stehen in unterschiedlichen Dateien zur Verfügung. Dabei
sind sie (teils ausschließlich) in *.conf dateien oder in recipes anwendbar.

\begin{description}
    \item[Standard operatoren wie +=, ?=, \ldots] Eine Auflistung der standard Operatoren ist zu finden unter:
    \cite[Seite 160]{Gonzalez2018:Embedded_Linux_Development_Using_Yocto_Project_Cookbook_2nd}
    \item[\_append, \_prepena, \_removed] Wird übergehend in *.conf Dateien
        verwendet. Genau wie die Standardoperatoren erweitern Sie eine Variable.
        Sie werden jedoch zu einem späeren Zeitpunkt als die standard Operatoren
        verarbeitet. Weiteres ist zu finden unter:
    \cite[Seite 160]{Gonzalez2018:Embedded_Linux_Development_Using_Yocto_Project_Cookbook_2nd}
    \item[INHERRIT] Genutzt ausschließlich in *.conf Dateien zun erben
        (includieren) von configurations Klassen.
    \item[inherrit] Genutzt ausschließlich in recipes *.bb und *.bbappend zum
        erben (inkludieren) von Bitbake-Classen.
    \item[include] Wenn Datei vorhanden, dann include ihren Inhalt.
        Ansonsten ignoriere den Befehl. \textit{include} lasst sich sowohl in
        configurations-Dateien als auch recepes verwenden um alle andren Arten
        von Dateien zu inkludieren (andere Recipes, Configs, \ldots )
    \item[require] Wie include nur das die Datei vorhanden sein muss, ansonsten
        beendet bitbake mit einem error.
\end{description}






\section{Yocto Tools}%
\label{sec:yocto_tools}

    Zum Arbeiten an Bitbake recipes, meta-layern, bitbake classes,
        bitbake Paketgruppen oder an Konfigurationsdateien stehen verschiedene
Werkzeuge beteit.
        \begin{itemize}
            \item \textit{bitbake-layers} help
            \item \textit{yocto-layers } help
            \item \textit{recipe-tool} help
            \item \textit{devtool } --help
            \item \textit{bitbake -c devshell <recipe>}
            \item \textit{bitbake -c devpyshell <recipe>}
            \item GNU Tools wie \textit{grep, awk, set, diff, cp, vim, usw}
        \end{itemize}
         Dabei können gleiche Aufgaben mit verschiedenen Werkzeugen oder auch
         manuel bearbeitet werden können und zum selben ergebnis führen.\\

         Nicht jedes Tool ist dabei für jeden \textit{Workflow} geeignet. Zum
         Teil ist es einfacher und weniger Fehleranfälliger einzelne Schritte manuell
durchzuführen. Etwa das Integrieren neuer Software\-komponenten in ein Image.





\section{Entwicklungsrollen und Workflows}%
\label{sec:workflows}
Generell ist das Arbeiten an einer Linux Distribution und das Arbeiten mit
Bitbake in verschiedene Aufgabenpakete zu unterteilen:
\begin{itemize}
    \item Anpassen eines Linux Kernels und Entwicklung neuer Kernel Module
    \item Anpassen von Hardware-Beschreibungen wie Device Trees und Bootloader
    \item Anpassen von meta-layern auf u.a. auf BSP und Applikationsschicht.
        Sowie erstellen von neuen meta-layern, recipes, bitbake-paketen, \ldots
    \item Entwickeln von Softwarekomponenten; Hardwarenah; nutzen von
        Systemcalls
    \item Entwickeln von Softwarekomponenten; Abstract; nutzen von höheren
        Funktionsbibliotheken
    \item Einbinden und patchen / erweitern von Softwarepakete in den Buildprozess,
\end{itemize}

Je nach Aufgabengebiet unterscheidet sich auch die Arbeitsweise:
\begin{itemize}
    \item Externe Software- oder Modulentwicklung, etwa in \textit{eclipse} oder
        \textit{qt-creator} sowie externes Testen der Komponenten. Bitbake kann
        für diese Aufgabe ein \acl{SDK} bereitstellen, welches die Cross-
        Entwicklung für die Ziel\-plattform ermöglicht. Siehe hierzu
        \fullref{cha:sdk}.
    \item Patchen oder erweitern von Software\-komponenten die Bereits in den
        Linux Kernel und Bitbake integriert sind, lassen sie am einfachsten durch
        Tool bewerkstelligen, die von der Yocto / OpenEmbedded Community bereit
        gestellt werden. Hierzu zählt z.B. das s.g. \textit{devtool}. Siehe
        hierzu \fullref{cha:yocto_devtool}. Aber auch eine temporäre
        Bitbake-bash oder bitbake-python \textit{development-shell} steht bei
        einfachen Anpassungen zu verfügung.
    \item Beim Arbeiten an recepis und meta-layern und somit dem erweitern und
        pflegen des Buildssystems, können helfen:
            \item \textit{bitbake-layers} help
            \item \textit{yocto-layers } help
            \item \textit{recipe-tool} help
            \item \textit{devtool } --help

\end{itemize}


