

\chapter{Bitbake}%
\label{cha:bitbake}

\section{Bitbake Buildprozess}%
\label{sec:bitbake_buildprozess}



Die Buildumgebung \textit{Bitbake}, im Kern bestehend aus Phython- und
Shellscripen die mit unterschiedlichen Softwareentwicklungstools wie GIT, Make
oder Autotools interagieren, führt in einer geregelten Reihenfolge verschiedene
Aufgaben aus. Die Abbildung auf \cite[S. 20]{Gonzalez2018:Embedded_Linux_Development_Using_Yocto_Project_Cookbook_2nd}
zeigt abstract den Build\-process:

\begin{itemize}
    \item Rhein\-folge der gepassten config files gepasst werden.
    \item Parsen der config files.
    \item Buildschritte (Siehe \ref{sec:bitbake_build_tasks} Seite \pageref{sec:bitbake_build_tasks}
    \item Packetieren in unterschiedliche Pakettypen
    \item Generieren und bereitstellen von \acfpl{SDK}
    \item Aufräum- und Nacharbeiten
\end{itemize}

\section{Bitbake build tasks}%
\label{sec:bitbake_build_tasks}

Jedes \textit{recipe} erbt eine Reihe von standard Build-Tasks.  Hierzu
gehören u.a.
\begin{itemize}
    \item Herunterladen (fetchen) und Sammeln von Source Dateien
        (z.B. Git Repositorien, ftp Servern oder lokalen Dateien.)
    \item Übersetzten, konfigurieren, patchen, installieren, verifizieren usw.
        von Paketen auf mehreren Prozessor Kernen
    \item Bereitstellen von Entwicklung und Verwaltungstools
\end{itemize}

Eine Auflistung der Standard-Task kann, mitsamt kurzer Beschreibung,
\cite[S. 171-172]{Gonzalez2018:Embedded_Linux_Development_Using_Yocto_Project_Cookbook_2nd}
entnommen werden.

\subsection{Fatch Task}%
\label{sub:fatch_task}
Eine der Hauptaufgaben von Bitbake besteht dadrinne, die jeweiligen
Softwarepakte in Ihren benötigten Versionen und Revisionsständen zusammen aus
unterschidlichen Quellen zusammen zu sammeln und dem Buildsystem zur Verfügung
zu stellen. Solche Quellen können sein:

\begin{itemize}
    \item Lokaler Bitbake-Download ordner/cache. Er enthält bereits einmal
        heruntergeladene Softwarepakete in jeweiligen Revisionständen.
    \item Lokaler Pfad zu Quelldateien; z.b. in einem eclipse oder QT Work\space
    \item Lokale Repository (z.B. lokales GIT Repository)
    \item Netz\-werkpfad zu einem Client oder Server im lokalen Netzwerk. Z.B.
        über Freigaben oder einem lokalen FTP Server
    \item Online Repository oder TFTP Server.
    \item Alternatives lokales oder Remote (online) Repository.
\end{itemize}

Die Reihe\-folge in welcher nach Source-Dateien gesucht werden soll ist zum
einen definiert durch:

\begin{itemize}
    \item das Recipe selbst welches das Software\-pakete innerhalb Bitbake bauen
        soll,
    \item durch Konfigurationsdateien wie \textit{./conf/local.conf),
    \tem sowie durch eine allgemein fest vorgegeben Reihe\-folge innerhalb
    bitbakes. (Siehe \cite[S.53]{Gonzalez2018:Embedded_Linux_Development_Using_Yocto_Project_Cookbook_2nd}
\end{itemize}



\section{Configurations-Dateien *.conf}%
\label{sec:configurations_dateien_conf}
Bitbake, sowie die Buildprozesse der einzelnen Recipes
werden gesteuert durch unterscheidliche \textbf{recipe-lokale} und
\textbf{globale} configurationsdateien.



