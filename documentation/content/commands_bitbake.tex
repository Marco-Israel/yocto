

################################################################################
########## Nice to know ########################################################
- An 'image' wich will be build is also described by recipes. So all commands
  you find next can be used for an image to build as well as for a recipes




################################################################################
########## Commands ############################################################

# create a Yocto (default) layer interactivly and add it to /build/bblayer.conf
# global active/used project layer configuration. You don't have do write 
# "meta" in front of"
yokto-layer create <myLayer>
bitbake-layers create-layer <mylayer>
bitbake-layers add-layer <layername>




#show all project layer
bitbake-layers show-layers



# show all recipes and the location layer it belongs to. You can grep for a kind
# of recipes type. 
bitbake-layer show-recipes [<recipes>]
bitbake-layer show-recipes [<recipes>] | grep image


#show all tasks for a recipes
bitbake -c listtasks <recipes>


#run Command <cmd>. Commands/tasks depend on the recipes.
bitbake -c listtasks <recipes>
bitbake -c <cmd> <recipes>


#print a bitbake/yokto environment variable (used and set inside the project)
bitbake -e <recipes> | grep ^<ENVVARIABLE>



#compile the recipes and jump into its temporary "working directory". Either
# by a normal development (bash) shell or a python shell.
bitbake -c devshell <recipes>
bitbake -c devpyshell <recipes>


#get the linux kernel version and provider used in a project
bitbake -e virtual/kernel | grep ^SRC_URI=
bitbake virtual/kernel -e | grep  "PREFERRED_PROVIDER_virtual/kernel"







