
\chapter{Ausblick}%
\label{cha:ausblick}




\section{Nächste Schritte}%
\label{sec:naeste_schritte}

\subsection{Entwicklungswerkzeuge als Docker Container}%
\label{sub:developmenthost}

Es wäre sinnvoll die Entwicklungstools in einem oder mehrere Docker
container vorkonfiguiert zur verfügug zu stellen. Hierzu zählen unter anderem:

\begin{description}
    \item[Eclipse] Vorkonfiguriertes Eclipse inclusieve Plugins, Cross-Compile
        und Remote debugging Einstellungen.
    \item[QT5] Vorkonfiguiertes QT mit Crosscompile und Remot Debugging
    \item TFTboot und NFSROOT Server in einem Container vorkonfiguriert bereitstellen
\end{description}


\subsection{Erweiterung der Scripte}%
\label{sub:erweiterung_der_scripte}

Das \textit{run.sh} script sollte so erweitert werden, das es \glq post\grq
oder \glq pre\grq Aufgaben vor oder nach dem aufrufen der \textit{dockerjobs.sh}
durchführt oder andere postbuild oder prebuild scripte aufruft. Denkbar wären:

\begin{itemize}
    \item Kopieren des zImages und \ac{DTB} in das \textit{TFT-boot} Verzeichnis
    \item Kopieren und extrahieren des rootfs in das nfs-rootfs Verzeichnis
\end{itemize}


Das \textit{run.sh} script erzeugt das \textit{dockerjobs.sh} Script sollte dem
run.sh script parameter übergeben werden. Anschließend startet das run.sh script
den Docker container in definierter version (gesetzt über Parameter oder direkt
innerhalb des run.sh scripts). Das dockerjobs.sh wird innerhalb docker durch das
image aufgerufen und enthält alle aufgaben welche durch den Container in
batchmode erfüllt werden sollen. Das dockerjobs.sh script lässt sich manuell
erweitern / erstellen. Es wird nur überschrieben, wenn dem run.sh
Ausführungs\-befehle übergeben werden.



\section{Security}%
Gerade zu Beginn der Entwicklung bietet sich an, zunächst auf viele
Sicherheit\-funktionen zu verzichten, da der gesamte Entwicklungs\-prozess
bereits komplex ist und ausreichend potentielle Fehlerquellen besitzt.
\\

Dennoch ist mindestens zum Ende eines Projektes, vor Veröffentlichung, das
Sicherheits\-konzept überarbeitet werden.
So müssen beispielsweise nachfolgende Themen bewertet und bearbeitet werden
\label{sec:security}

\begin{itemize}
    \item Linux Kernel härten
    \item SELinux
    \item Gesamt System härten
    \item SMACK
    \item Benutzer, Passwörter, Zugrifffsrechte, ACLs
    \item Netzwerkschnittstellen und Kommunikation absichern. Beispielweise
        durch verschlüsselte Datenübertragung
    \item Debugging, Flashing, Tracing Schnittstellen entfernen oder
        einschränken
\end{itemize}


\section{Lizenzen}%
\label{sec:lizenzen}

Eine wichtiges Thema ist die Lizensierung neuer Softwarekomponenten, welche
zusammen mit Open-Source paketen (i.d.R. mindestens dem Linux Kernel) genutzt
werden und mit diesen Kompatible sein muss. So muss ich über folgende
Kombinationen gedanken gemacht werden.

Lizenverwaltung und Kompatibilität von:
\begin{itemize}
    \item Neuen Sofwarekomponenten
    \item Bestehenden  Sofwarekomponenten
    \item Verwendeten / Eingebundenen Softwarepaketen, z.B. über genutzte
        meta-datan Layer aus Community Quellen.
\end{itemize}











